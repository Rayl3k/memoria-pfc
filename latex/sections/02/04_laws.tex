\section{Les lleis reguladores}

    \paragraph{}
    Les seccions anteriors han introduït i descrit les ocupacions principals de la genealogia en els àmbits professional i amateur, així com el paper d'aquesta en la història. També s'ha esmentat que moltes de les dades amb les quals aquesta ciència interactua són de caràcter personal i per tant, sensibles a un ús impropi si no són regulades i protegides sota certes circumstàncies.

    És per aquest motiu que bona part de les dades públiques enregistrades per l’estat, sobretot aquelles que afecten a persones que encara són vives, es troben regulades sota un conjunt de lleis i legislacions. Aquestes lleis varien de nació en nació i per tant, no existeix un estàndard de quina informació és accessible pel domini públic, quina no i sota quines circumstàncies aquesta informació pot ser accedida.

    En el cas de l’estat espanyol són dues les principals lleis que regulen l’accés a les dades genealògiques. La \gls{LOPD}\footfullcite{lopd} i la legislació consolidada: Llei 20/2011, del 21 de juliol del Registre Civil\footfullcite{spainCivilRegistry}.

    El registre civil espanyol conté informació detallada d'una persona, relacionada amb el seu naixement, relacions d’ascendència i descendència, nom i cognoms, emancipació, declaracions de concurs o suspensió de pagaments, nacionalitat, etcètera, etcètera.  Com podem veure, aquest registre conté tota mena d’informació sensible i al mateix temps, de gran valor de cara a estudis genealògics.

    Els habitants d'Espanya podem demanar accés a l'entrada del registre civil d'una persona mitjançant la presentació d'una sol·licitud digital, escrita o presencial. Per aconseguir aquesta informació caldrà proporcionar tan dades personals pròpies com el motiu pel qual es vol accedir a la partida en concret. Motius recurrents són l'estudi genealògic, gestions administratives o simplement, la recaptació d'informació.

    Per altra banda, accedir al gruix de la informació no és fàcil i els genealogistes porten xocant amb portes tancades des de fa molts anys. Relacionat amb aquest aspecte, durant l'any 2011, es va aprovar una nova llei del Registre Civil que tenia com a objectiu racionalitzar l'estructura del registre i desjudicialitzar-lo. Aquesta llei havia d'entrar en vigor a partir del 2014, data que va ser posposada fins al juliol del 2015 i que recentment ha tornat a ser ajornada fins al 30 de juny del 2017.

    La part que farà referència sobre si el nou registre contemplarà l'accés al públic de cara a la recerca genealògica encara està a l'aire, però sembla que hi ha certa esperança gràcies a la inclusió del següent apartat en l'article 80:

    \begin{displayquote}
        4. Amb caràcter excepcional i amb finalitats d'investigació familiar, històrica o científica, es podrà autoritzar l'accés a la informació registral en els termes que reglamentàriament s'estableixin.
    \end{displayquote}

    Així doncs, sembla que un futur no molt llunyà, aquesta informació podria passar a ser explotable a gran escala de cara a estudis familiars, històrics o científics. Això sí, sempre respectant la \gls{LOPD}.
