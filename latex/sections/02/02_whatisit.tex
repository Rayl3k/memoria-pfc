\section{Què és la genealogia?}

    \paragraph{}
    La \gls{genealogia}\footfullcite{wiki:genealogy} (del grec:  \emph{'genea', 'generació'}; i, \emph{'logos', 'coneixement'}) és també coneguda pel nom d’història familiar. Aquesta ciència consisteix en l’estudi de les famílies, el seguiment dels seus llinatges tant ascendents com descendents i l’estudi de la història de les persones.

    Els genealogistes, o persones dedicades a la genealogia, tant en l’àmbit privat com personal, utilitzen com a recurs d’investigació arxius històrics rics en dades. Exemples d'aquests recursos poden ser les partides de naixement, documents de defunció, registres d’emigració o altres documents informatius del mateix caire.  L'objectiu d'aquests documents és obtenir informació sobre una persona o família per així poder demostrar relacions de parentesc i llinatge o bé, fets empírics relatius a la vida d'un individu en concret.

    Un altre recurs que es veu cada cop més utilitzat és l'anàlisi genètic, mètode que té una rebuda, demanda i interès més elevat, en l’àmbit personal que no pas en el científic. La principal finalitat d'aquest mètode és esbrinar relacions familiars passades i presents de l'individu a través de l'anàlisi dels seus gens.

    Els motius pels quals una persona pot estar interessada a endinsar-se en el món de la genealogia són diversos. Un exemple podria ser el desig de situar la seva família en un marc més ampli dins de la història o bé, el sentiment de responsabilitat de cara a preservar la història familiar per les futures generacions.

    Els aficionats a la genealogia, que la practiquen com a hobby, generalment investiguen la seva ascendència o la d’una persona propera. Per altra banda, els pro\-fe\-ssio\-nals, acostumen a encarregar-se de realitzar recerques genealògiques per tercers, estudiar i ensenyar mètodes de recerca o mantenir les seves pròpies bases de dades.

    Cal entendre que la genealogia no tracta només de recopilar informació sobre el moment històric en què una persona va néixer, viure o morir, sinó també recollir informació sobre l'estil de vida que aquella persona va portar, les seves biografies o quins van ser els esdeveniments i motivacions que van conduir i marcar la seva existència. En altres paraules, podríem dir que una part de les preguntes que la genealogia pretén respondre és la de com van viure o quin caràcter van mostrar els nostres avantpassats davant els esdeveniments històrics que van marcar la seva època, com per exemple, la segona guerra mundial.

    Voldríem tancar aquesta secció indicant que si l'interès per la genealogia, ha anat en augment en els últims temps, és en gran part gràcies a la digitalització de documents, fet que ha permès que genealogistes amateurs disposin d’un ventall d’eines molt superior al que van disposar els seus avantpassats i per tant, que les possibilitats de mantenir un arbre familiar o realitzar recerca genealògica quedin a l'abast de tothom.
