\subsubsection{Iniciant la cerca de persones}

\paragraph{}
Quan l'usuari prem el botó de cerca, la primera tasca del controlador és validar que els continguts introduïts en el formulari són correctes. En cas que el formulari no compleixi les condicions de cerca, es dispararà un conjunt d'errors informant l'usuari de la informació a corregir i no es llençarà cap crida contra l'API.

Per realitzar aquesta validació, el formulari escapa cada camp que pot ser introduït per l'usuari i comprova si s'ha de realitzar alguna verificació específica del contingut.

Escapar el camp d'un formulari significa codificar certs caràcters especials, com per exemple, `\&', `<', `>', `/', etcètera, per tal d'assegurar el seu correcte transport pels URL i evitar atacs malintencionats, que pretenguin trencar el HTML i afectar al servidor, mitjançant injeccions de codi.

El fitxer, \emph{formValidation.js}, s'encarrega de realitzar aquestes transformacions i validacions. Mostrem a continuació, el codi que escapa els caràcters no desitjats. Per escapar un paràmetre, només cal enviar-lo contra la funció \emph{escapeHtml()}.

\begin{lstlisting}[style=rawOwn,caption={Funcio \emph{escapeHtml()} i la variable \emph{entityMap}}]
var entityMap={`&':`&amp;', `<':`&lt;', ... `/': `&#x2F;'};

function escapeHtml(string) {
    return String(string).replace(/[&<>"'\/]/g, function (s) {
        return entityMap[s];
});
}
\end{lstlisting}

Si no es mostra cap error de validació, significa que el formulari és correcte i que compleix amb les condicions de cerca.

Arribats a aquest punt, es prepara l'objecte \emph{params}, que emmagatzema tots els paràmetres de cerca descrits a l'apartat~\ref{sec:searchParams}, s'eliminen de la crida aquells pa\-rà\-me\-tres que no han estat introduïts per l'usuari i es demana la cerca dels primers quinze resultats mitjançant la crida a la funció \emph{print\-Persons\-To\-Table(0)}.
