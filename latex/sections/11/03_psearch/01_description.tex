\subsection{Descripció de la funcionalitat}\label{sec:searchTree}

    \paragraph{}
    La funcionalitat d'exemple de cerca, permet buscar instàncies de persones per tot l'arbre familiar de FamilySearch.

    La funcionalitat ha estat habilitada per permetre la utilització de tots els paràmetres de cerca disponibles per l'API i el SDK, descrits a la cinquena secció de la memòria. De totes maneres, per alguna raó desconeguda, sembla que la cerca d'esdeveniments, relacionats a familiars de la persona cercada, no funciona.

    En concret, cercar pels llocs o dates concretes de naixement, defunció o casament, dels relatius més propers, acostuma a traduir-se amb l'obtenció de zero resultats. Curiosament, FamilySearch tampoc ofereix aquests paràmetres de cerca en la seva pàgina web. Per tant, tot i conèixer els problemes amb aquests camps, hem decidit deixar-los actius perquè els usuaris puguin experimentar amb ells i observar-ne les limitacions.

    La cerca de persones es realitzada amb el paràmetre \emph{count}, que indica el nombre de resultats a retornar, sense especificar. En aquesta situació, l'API retorna quinze persones per defecte i un cop retornats els resultats, el controlador pinta la informació d'aquestes quinze persones en una taula. Aquesta taula recull, per cada persona, informació sobre el seu identificador, nom, data de naixement i data de defunció.

    La taula de resultats permet la navegació entre totes les persones que complien les condicions de cerca, carregant-les en blocs de quinze persones. Cal tenir en compte, que per cada acció de paginació, el client ha de realitzar una crida a l'API de FamilySearch i que per tant, pot tardar alguns segons a recarregar-se amb nova informació.

    Finalment, la taula de resultats, permet la selecció d'una de les persones, per tal de desplegar tota la informació principal que es disposa sobre ella, els seus relatius més propers i informació s'obre l'ascendència i descendència de la persona.
