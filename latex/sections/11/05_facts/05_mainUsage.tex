\subsection{Principal interés d'ús}

    \paragraph{}
    Com hem comentat en la descripció d'aquesta funcionalitat, a pesar de ser de les més simples que hem implementat, podia ser la que tingués un interès més elevat, un cop obtingut l'accés a les dades de producció.

    La funcionalitat va ser programada per relacionar fets històrics, com la gran re\-cessió del vint-i-nou o la segona guerra mundial, amb les taxes de natalitat i defuncions.

    Un objectiu secundari d'aquesta funcionalitat, era intentar comprendre si les bases de dades de FamilySearch, representaven una fotografia de la realitat, donada l'existència d'un cert nombre de registres o com de desviades es trobaven.

    Per exemple, tal com s’ha exposat en una proposta de projecte, és conegut que durant la segona guerra mundial, el nombre de matrimonis es va disparar als Estats Units. Sobre el paper, aquesta funcionalitat hauria de permetre realitzar una primera comprovació, sobre si aquestes relacions conegudes són també observables en les dades de l'organització.

    Desafortunadament, sembla que la funcionalitat de cerca no acaba de comportar-se com ens esperàvem i que les dades de producció, no són accessibles al nivell que crèiem. Això no vol dir que les propostes de projecte relacionades a aquesta funcionalitat, no puguin ser realitzades, sinó que caldrà buscar mètodes alternatius, per tal d'explorar-les.

    En el segon apartat de la següent secció de la memòria, exposarem la restricció que ens hem trobat, en jugar amb les dades de producció.
