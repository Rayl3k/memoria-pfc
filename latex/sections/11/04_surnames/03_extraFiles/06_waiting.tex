\subsubsection{Progressió de la cerca}

\paragraph{}
Un dels aspectes d'usabilitat més interessants d'aquesta funcionalitat és la secció dedicada a mostrar la progressió de la cerca. Si ja havíem comentat que a mesura que es rebien les dades dels diferents anys, aquestes són representades en els gràfics pertinents, aquesta secció de la pàgina compleix un rol similar, però més informatiu.

En el moment que es llança una cerca, l'usuari és transportat a aquesta secció. Durant el temps que dura la cerca, aquesta secció mostra el cognom pel qual s'està realitzant la búsqueda, la duració estimada (calculada mitjançant la fórmula: \emph{númeroPaïsos} $\times$ \emph{númeroAnys} $\times$ \emph{apiDELAY}), la informació sobre el país i any, dels quals s'està esperant la resposta per part del SDK i una barra del progrés actual, respecta el total estimat.

Una cosa que cal tenir en compte és que les crides al SDK són asíncrones i que per tant, no està garantit que el retorn d'aquestes segueixi el mateix ordre que l'ordre d'enviament. Això significa, que potser, el bloc HTML que mostra la progressió de la cerca i indica que s’estan esperant dades per un país i any concret, pot no representar la realitat. De totes maneres, s’espera que en la gran majoria dels casos, hi hagi una correlació perfecta entre l’ordre d’enviament i el retorn de les peticions des del SDK.

En qualsevol cas, el fet més important relatiu a la barra de progrés, és que aquesta arribi al 100\% quan s'han processat totes les crides a l’API i que cada crida emmagatzemi el resultat, a les caselles de la matriu que li pertoquen. Recordem que això és aconseguit, gràcies als paràmetres \emph{i} i \emph{k}, encapsulats a les crides.

Quan la cerca és completada, l'estat dels components de la secció canvia i s'indica que la cerca ha estat completada pel cognom especificat, s'elimina el temps estimat de finalització i és substituït per una indicació sobre la localització dels resultats i es mostra també el nombre total de anys i països, pels quals s'ha efectuat la cerca. L'efecte de moviment en la barra de progressió, també és eliminat per evitar confondre a l'usuari.

La imatge~\ref{fig:waitingSurnames} mostra dos estats diferents, de la secció progressió de la cerca.

\begin{figure}[h]
    \includegraphics[width=\linewidth]{11/03_surnamesSearch/06_waitingDesktop2}
    \includegraphics[width=\linewidth]{11/03_surnamesSearch/07_waitingDesktopComplete2}
    \centering
    \caption{Exemples de la secció progressió de la cerca}\label{fig:waitingSurnames}
\end{figure}
