\subsection{Recomanacions d'utilització}

    \paragraph{}
    Volem recordar en aquesta secció, que les crides al SDK de FamilySearch són asíncrones i que per tant, res ens impedeix realitzar deu, cent o mil crides simultànies al SDK i en conseqüència, a l’API de FamilySearch.

    Tanmateix, si recordem la funcionalitat de `throttling' introduïda anteriorment en aquesta memòria, aquesta impedeix l’abús de la connexió i en cas de fer-ho, les nostres peticions quedarien bloquejades.

    Com que la funcionalitat evolució geogràfica d’un cognom, llença una cerca al SDK per cada país i any de l’interval, s'ha imposat en el codi, de forma manual, una separació de dos segons entre les diferents crides. Aquesta restricció, fa que el temps d'execució aproximat per una cerca, sigui de: 2 $\times$ nombre de països $\times$ nombre d'anys (segons).

    Per aquest motiu, es recomana als usuaris mantenir un nombre baix d'anys i països en les seves cerques de prova. Recordar també, que no tots els països disposen del mateix nombre de registres i que per tant, algunes cerques són candidates a no obtenir cap resultat.
