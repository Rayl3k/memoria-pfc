\subsection{Les tecnologies Javascript i jQuery}

    \paragraph{}
    Tot i que aquestes dues tecnologies no són exactament el mateix, volem presentar-les de forma conjunta, ja que són utilitzades d’aquesta forma en el món de les aplicacions web.

    Javascript\footfullcite{wiki:javascript} és un llenguatge dinàmic d'alt nivell que s'ha convertit, conjuntament amb el HTML i el CSS, en un dels tres pilars de la programació web. Gairebé totes les pàgines l'utilitzen i és suportat per tots els navegadors moderns. Javascript és un llenguatge de programació molt flexible i permet diferents estils de programació, com pot ser la programació orientada a objectes o estils de programació imperatius i funcionals.

    Cal comprendre que Javascript també és utilitzat fora del món web, però aquest  representa el seu marcat principal. Tampoc s'ha de confondre aquest llenguatge amb el llenguatge de programació Java. A pesar de les similituds entre els dos, es tracta de dos llenguatges de programació diferents amb desenvolupaments separats.

    Per altra banda, jQuery\footfullcite{jQuery} és una llibreria de Javascript plena de funcionalitats, dedicada a la manipulació de documents HTML, CSS i gestió de les comunicacions entre aquests i el servidor de l’aplicació web. Aquesta llibreria, s'ha convertit en un estàndard de la programació web i resulta indispensable de cara a crear aplicacions web funcionals i interactives.

    jQuery destaca sobretot per la possibilitat d’executar blocs de codi diferents quan l’usuari interactua de forma específica amb elements del HTML que disposen de certes etiquetes de classe o identificadores, de la mateixa forma que els fitxers CSS utilitzen aquestes etiquetes per definir l’aparença dels components HTML de la web.

    Javascript i jQuery són utilitzats de forma conjunta en la capa anomenada `back-end del front-end’ i seran, per tant, els encarregats de realitzar la funció de Controlador en l'arquitectura MVC de la nostra aplicació web.

    No volem entrar gaire més en detall de com s’utilitzen aquestes tecnologies, ja que les possibilitats són realment il·limitades i intentar fer un petit manual d’ús, resultaria impossible. No obstant això, per tal que s'entengui una mica més la funcionalitat d'aquestes, imaginem-nos una pàgina web que disposa d'un botó amb identificador \emph{submit}, que un cop pressionat, canvia el valor d'un camp de text del HTML de la pàgina web.

    En aquest exemple, s'utilitzaria jQuery per detectar que el botó submit del HTML ha estat pressionat, Javascript per definir una nova variable de text i jQuery per imprimir el contingut d’aquesta nova variable al camp de text del HTML. El codi podria tenir un aspecte similar al que segueix:

\begin{lstlisting}[style=rawOwn,caption={Example of jQuery and Javascript}]
$(`#submit').click(function () {
	var newText = `next text to display';
	$(`#textField').text(newText);
});
\end{lstlisting}

    Tot i ser un exemple molt bàsic, creiem que pot ajudar a comprendre perquè s'utilitzen aquestes tecnologies i com interactuen amb els elements de la pàgina web.
