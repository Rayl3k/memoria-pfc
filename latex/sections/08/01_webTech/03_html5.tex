\subsection{La tecnologia HTML 5}

    \paragraph{}
    La primera tecnologia que calia estudiar, si es volia implementar una pàgina web, era la tecnologia HTML i més concretament, el HTML 5.

    Aquesta tecnologia és utilitzada per crear l’estructura de les aplicacions web i configurar quina informació s’ensenyarà a cada lloc del navegador. La tecnologia HTML, és considerada un llenguatge d’etiquetatge.

    El concepte llenguatge d'etiquetatge, significa que els diferents blocs de contingut, s'envolten per etiquetes d'obertura i clausura, que atorguen un significat concret al contingut situat a l'interior.

    Per exemple, per indicar la creació d’un bloc de contingut (<div>), que en el seu interior, conté tres paràgrafs (<p>), es podria utilitzar una estructura com la que segueix:

    \begin{lstlisting}[style=rawOwn,caption={Bloc de contingunts HTML amb tres paràgrafs}]
<div>
    <p> ... </p>
    <p> ... </p>
    <p> ... </p>
</div>
    \end{lstlisting}

    \paragraph{}
    El llenguatge HTML disposa d’un ampli ventall d’etiquetes diferents, que permeten introduir imatges, crear enllaços web, ressaltar el text en negreta o cursiva, etcètera, etcètera.

    A més a més, aquest llenguatge crea el que s’anomenen estructures jeràrquiques, on un element pot tenir el rol de pare o fill d’un altre. Per exemple, en l’exemple de codi mostrat, el contenidor <div> és el pare de tots els paràgrafs. El llenguatge HTML, permet crear tants nivells de jerarquia diferents com es desitgin.

    Així doncs, el llenguatge HTML és utilitzat per crear la capa d'estructures explicada a l'apartat: `Les tres capes del disseny web', i per tant, podem dir que aquesta tecnologia és utilitzada únicament,a la capa del front-end de la nostra aplicació web.
