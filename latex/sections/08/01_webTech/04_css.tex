\subsection{La tecnologia CSS}

    \paragraph{}
    La tecnologia CSS\footfullcite{wiki:css}, també coneguda com a generació de fulls d'estil en cascada, és un llenguatge utilitzat per explicitar com ha de ser l'aspecte i forma dels diversos elements que apareixen en l’estructura d’una pàgina web o el que és el mateix, en el seu HTML.

    El CSS va néixer per poder separar el contingut d'un document, de la presentació d'aquest. El llenguatge CSS permet, entre moltes altres coses, decidir la font i estil de cada element de la pàgina web, l'alineació del text, les separacions entre els diferents components del HTML, els colors o imatges de fons, l'estil dels enllaços web, les transicions entre els diferents estats d’un component HTML, etcètera, etcètera.

    Una de les principals característiques del CSS és que al tractar-se d'una fulla d'estils en cascada, un element pot tenir definides diferents regles que marquen el valor d’un cert atribut i el llenguatge CSS és capaç de sabers quina ha d’aplicar, segons la posició de cada regla, dins de l'estructura HTML.

    A menys que s’especifiqui de forma contrària, la regla que s’aplica per per\-so\-na\-lit\-zar l'atribut d'un element HTML, és la que aplica directament sobre l’element o la heretada del seu pare més proper, que té aquell atribut estilitzat.

    D'aquesta forma, s'aconsegueix dotar d'un comportament genèric a certs components HTML i personalitzar només aquells que volem dotar d'atributs diferents.

    En resum, la tecnologia CSS permet controlar tots els atributs que fan referència a l'aparença de les diferents estructures i components HTML. Això, ho aconsegueix mitjançant tres classes d'etiquetatge diferents:

    \begin{itemize}
        \item \textbf{Etiquetes HTML:}  Aquestes etiquetes s'utilitzen per aplicar regles a tots els elements HTML que encaixin amb una etiqueta en concret. Per exemple, si s'utilitza l'etiqueta `div', es podria decidir des d’aquesta regla la tipografia a ser utilitzada per tots els contenidors del HTML.
        \item \textbf{Etiquetes de classe:} Aquestes etiquetes són creades pel desenvolupador mitjançant la concatenació del caràcter `.', amb un nom qualsevol (per exemple: `.color-blue’). Com el seu nom indica, les etiquetes de clase permeten assignar regles d'estil a classes concretes. Aquestes classes, poden ser incloses dins dels components HTML, proporcionant així l'estil desitjat a només aquell conjunt de components marcats per l'etiqueta de classe especificada.
        \item \textbf{Etiquetes identificadores:} Aquestes etiquetes són similars a les de classe, però en comptes d'utilitzar el caràcter `.', abans del nom que vol ser introduït, s'utilitza el caràcter `\#'. La diferència principal entre les etiquetes de classe i les identificadores, és semàntica. En principi, en un fitxer HTML, hauria d’existir com a màxim una instància de cada etiqueta identificadora, mentre que les etiquetes de classe, poden ser utilitzades en múltiples elements HTML.
    \end{itemize}

    Per exemple, podríem incloure l’etiqueta identificadora \emph{first} i l’etiqueta de classe \emph{notícia} a un element HTML, de la següent forma:

    \begin{displayquote}
        <div id=`first’ class=`noticia’> ... </div>
    \end{displayquote}

    El llenguatge CSS també ofereix la possibilitat de definir regles més complexes, com per exemple, definir un conjunt d'atributs per aquells elements que tinguin un pare específic a l’estructura HTML o que es trobin afectats per esdeveniments especials, com podria ser per exemple, el `mouseover’ (element HTML enfocat pel cursor del ratolí).

    Tanmateix, les bases són les mateixes tant pels casos simples com pels casos més complexos i creiem que el text introduït en aquesta secció, hauria de ser suficient de cara a comprendre la funcionalitat d'aquesta tecnologia i fer-se una idea de com funciona.

    La tecnologia CSS es veu utilitzada únicament a la capa del front-end de l'aplicació web.
