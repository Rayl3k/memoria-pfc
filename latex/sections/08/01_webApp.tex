\section{Decisió del tipus d'aplicació a implementar}

    \paragraph{}
    Per poder començar a estudiar el conjunt de tecnologies que el projecte requeriria, necessitàvem saber quina mena d'aplicació seria implementada.

    Des del principi, teníem bastant clar, que de disposar de l’oportunitat, intentaríem implementar una pàgina web. Després de realitzar l’estudi inicial de l’API i de les diferents opcions de desenvolupament possibles, crear una pàgina web semblava l’opció més flexible i factible i per tant, vam decidir tirar endavant per aquest camí.

    Una pàgina web representava l’opció més factible i flexible, ja que els protocols per integrar-se amb APIs es troben bastant desenvolupats i a més a més, oferia l’oportunitat d’utilitzar tant els SDK oficials, com les diverses eines de desenvolupament, que facilitarien les tasques de creació i interacció amb usuaris.

    Per tots aquests motius, a part de la motivació personal d’assolir les habilitats necessàries per desenvolupar una aplicació web, aquesta opció semblava la més adequada.

    Així doncs, un cop decidit que s'implementaria una pàgina web, calia fer un reconeixement de les diferents tecnologies disponibles en el mercat i escollir-ne, les més adequades, que poguessin treballar de forma conjunta.

    Les tecnologies estudiades poden ser dividides en tres grans blocs: Les tecnologies per la creació d’aplicacions web, les tecnologies de desenvolupament i les tecnologies de desplegament al núvol.

    Les tecnologies per la creació d’aplicacions web, representen aquell conjunt de llenguatges, arquitectures i frameworks, que serien utilitzats de cara a la construcció de la pàgina web. Amb altres paraules, el conjunt d’eines i llenguatges que s’utilitzaria per implementar el servidor, les comunicacions entre el servidor i l’API de FamilySearch i el frontal o interfície d'usuari de l’aplicació.

    Les tecnologies de desenvolupament, representen el conjunt de tecnologies i eines específiques, que han estat utilitzades per assistir i facilitar, la creació de l’aplicació web.

    Finalment, les tecnologies de desplegament, fan referència a l'allotjament web escollit i les tecnologies necessàries per poder completar el desplegament de l’aplicació al núvol.
