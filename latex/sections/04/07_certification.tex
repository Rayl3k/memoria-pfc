\section{Procés de certificació}

    \paragraph{}
    Per tal de poder connectar les aplicacions desenvolupades per tercers al conjunt de dades oficial de FamilySearch, aquestes aplicacions han de ser sotmeses a un procés de certificació.

    Les aplicacions poden ser certificades per l'ús comercial o per ús limitat. Les Apps que volen ser certificades només per ús limitat, requereixen un anàlisis tècnic sobre el seu funcionament, mentre que de les aplicacions d'ús comercial també són analitzades des d'un punt de vista de negoci i marketing. Com a contrapartida, les aplicacions d'ús limitat no poden aparèixer a la galeria d'aplicacions.

    Hi ha tres tipus de certificacions principals:

    \begin{itemize}
        \item \textbf{Certificació de lectura:} L'aplicació ha de ser certificada per la lectura de tots aquells recursos als quals accedeix. També cal que compleixi amb certs estàndards del procés d'identificació.
        \item \textbf{Certificació d'escriptura:} En cas que l'aplicació realitzi operacions d'escriptura, aquestes també hauran de ser revisades per part de FamilySearch.
        \item \textbf{Certificació per transferència d'arxius en grans quantitats:} Per aquelles organitzacions que tinguin permís per accedir als protocols de transferència de dades en grans quantitats, cal certificar i revisar, conjuntament amb FamilySearch, les operacions realitzades contra aqueta API.
    \end{itemize}

    \paragraph{}
    Es veuran més detalls sobre el procés de certificació en la part pràctica de la memòria del projecte.

    Un cop una aplicació ha estat certificada, en cas que aquesta modifiqui les operacions de lectura o escriptura que realitza, caldrà tornar a certificar-la abans de desplegar els canvis a producció.

    Existeix un procés encarregat de controlar que es compleixen els estàndards marcats per FamilySearch i en cas de no complir-los, pot significant el retirament dels drets d'accés a producció.

    Per acabar aquesta secció, comentar que mantenir una relació formar amb FamilySearch, proporciona certs beneficis com aparèixer en les seves galeries d'aplicacions i poder utilitzar el logotip d'aplicació certificada, entre altres petits avantatges.
