\section{Diferents entorns de treball}

    \paragraph{}
    Es disposa de molt poca informació sobre els diferents entorns de treball referents a l'API de FamilySearch. Per aquest motiu, va fer falta realitzar una investigació a través de diferents codis i documentacions, per poder inferir, al final, l'existència de quatre entorns de treball diferents:

    \begin{itemize}
        \item \textbf{Sandbox:} Entorn de treball únic per cada usuari. Utilitzable durant el desenvolupament de noves aplicacions i que replica les funcionalitats de l'API sense tenir accés a les dades de producció. Es pot entendre com un entorn de proves.
        \item \textbf{Staging:} Entorn de treball que s'utilitza per validar que una aplicació desenvolupada en un sandbox, està realment preparada per accedir a les dades de producció. Actualment en desús.
        \item \textbf{Beta:} Entorn de treball sobre el qual l'API va desplegant noves versions. Quan es vol realitzar un canvi en les funcionalitats o estructura de l'API, s'utilitza aquest entorn perquè els desenvolupadors puguin testejar el funcionament de les seves aplicacions abans de desplegar la nova versió a producció.
        \item \textbf{Producció:} Entorn de treball que pot accedir a les dades oficials de FamilySearch. Totes les aplicacions aspiren a connectar-se a aquest entorn de treball.
    \end{itemize}
