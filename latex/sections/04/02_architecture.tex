\section{L'arquitectura de l'\gls{API}}


    \subsection{Què és una \gls{API}?}

    \paragraph{}
    Abans d'entrar en detall en com funciona una \gls{API}, estaria bé definir, amb una mica més de precisió, en què consisteix exactament.

    Una \gls{API}, de l'anglès `Application Programming Interface' o Interfície de programació d'aplicacions en català, representa el conjunt de subrutines, funcions i procediments que ofereix una biblioteca per tal de ser utilitzada en el software de tercers com una capa d'abstracció. Aquest conjunt de subrutines, funcions i procediments, acostumen a oferir accés a certs serveis o conjunts de dades d'un particular, a tercers, de forma controlada.


    \subsection{L'arquitectura REST}

    \paragraph{}
    L’arquitectura sobre la qual està creada l’\gls{API} de FamilySearch és una arquitectura REST. Les sigles provenen de l’anglès i representen el concepte: \gls{REST}.

    Les arquitectures \gls{REST} es caracteritzen per estar orientades els recursos més que a les accions que es poden realitzar sobre ells i com a peculiaritat, es caracteritzen per sis regles o restriccions.

    Aquestes són: interfície uniforme, sense estat, client-servidor, emmagatzemables, sistema per capes, i codi sota petició. Aquests sis conceptes són doncs els que defineixen les bases de es arquitectures \gls{REST}.

    És diu que una arquitectura \gls{REST} és orientada als recursos perquè estan construïdes al voltant d’objectes i les relacions entre aquests, en comptes d’accions. Per exemple, a l’\gls{API} de FamilySearch, es parla de persones i esdeveniments, en comptes de llegir persones o crear esdeveniments i en canvi, aquestes operacions, passen a formar part dels objectes \emph{Persona} i \emph{Esdeveniment.}

    L'intercanvi de dades es produeix mitjançant l'ús de diferents representacions. Aquestes, expliquen com els recursos són tractats per l’\gls{API} i de quina forma han de ser realitzades les comunicacions entre el servidor i el client. Els formats més freqüents són JSON i XML. FamilySearch ofereix suport per ambdós formats.

    Per posar un exemple reduït que il·lustri el que estem explicant, podem descriure de la següent forma, el que podria ser una operació contra l'\gls{API} de FamilySearch, especificant quin element seria considerat el Recurs, quin el Servei i quin la Representació:

    \begin{itemize}
        \item \textbf{Recurs:} Persona (informació relacionada amb una persona en concret)
        \item \textbf{Servei:} Obtenir informació de la persona (GET)
        \item \textbf{Representació:} Nom, cognoms, esdeveniments relacionats amb la vida de la persona, etcètera, en format \gls{XML} o \gls{JSON}.
    \end{itemize}

    \paragraph{}
    Com també s'ha comentat, l’arquitectura \gls{REST} es caracteritza per la implementació de sis restriccions imposades sobre el sistema. A continuació s'exposa amb més detall, cada una d'elles.


    \subsubsection{Interfície uniforme (uniform interface)}

    \paragraph{}
    Aquesta restricció s'encarrega de definir la interfície de comunicació entre el client i el servidor.

    En una arquitectura REST, s'utilitzen els protocols de comunicació \gls{HTTP} i \gls{HTTPS} de forma conjunta amb els \gls{URI}, per aconseguir accés als diferents recursos i operacions proporcionades per l'\gls{API}.

    Els verbs permesos pels protocols de comunicació web són els coneguts: get, put, post, delete, options and head.

    Per exemple, per fer la petició de lectura sobre el recurs d’una Persona a l’\gls{API} de FamilySearch, executaríem la següent crida \gls{HTTP} o \gls{HTTPS} mitjançant el verb i URI especificats a continuació:\\ \verb|GET /platform/genealogies/persons/2:2:PPPJ-MYZ7|.


    \subsubsection{Sense estat (stateless)}

    \paragraph{}
    Aquesta restricció implica que el servidor no emmagatzema la informació del client. Això implica que cada petició d'aquest cap a l’\gls{API}, ha de contenir tota la informació necessària perquè el servidor l'identifiqui i es defineixin les regles de comunicació adequades per processar la petició.

    Hi ha exemples d'operacions a l'\gls{API} de FamilySearch, com per exemple el procés d'identificació Oauth V2, que no són realment RESTful, doncs aquestes sí que guarden informació del client durant les diferents parts de la comunicació.


    \subsubsection{Client-servidor (client-server)}

    \paragraph{}
    Per comprendre aquesta restricció, cal comprendre primer en què consisteix un sistema desconnectat.

    En el cas de les arquitectures REST, un sistema desconnectat implica que el client mai tindrà accés directe a les bases de dades que emmagatzemen la informació, i que per tant, sempre haurà d'accedir a les dades mitjançant l’intermediari. En aquest cas, l'\gls{API}.

    Els protocols de comunicació descrits prèviament (\gls{HTTP} i \gls{HTTPS}), i la interfície de comunicació, són els encarregats de gestionar les comunicacions entre client i servidor.


    \subsubsection{Emmagatzematge en el client (cacheable)}

    \paragraph{}
    Aquesta restricció fa referència a si les respostes retornades, des del servidor, al client, poden ser emmagatzemades per aquest i durant quant de temps les pot guardar. Existeixen tres nivells de configuració diferents:

    \begin{itemize}
        \item \textbf{Implícit:} Si és el client el que decideix quant de temps guardarà les dades o informació retornada, es tracte d'emmagatzematge implícit.
        \item \textbf{Explícit:} Si és el servidor el que mana i posa les regles, parlem d'emmagatzematge explícit.
        \item \textbf{Negociat:} Quan el client i el servidor negocien i arriben a un acord, es tractea d'emmagatzematge negociat.
    \end{itemize}


    \subsubsection{Sistema per capes (layered system)}

    \paragraph{}
    Aquest principi, o restricció, es basa en el fet que el client no pot assumir que tindrà connexió directa amb el servidor. És a dir, poden existir diferents intermediaris en forma de hardware i software entre client i servidor.

    Això, facilita l’escalabilitat i persistència del sistema gràcies al fet que el client no s’ha de preocupar de comunicar-se amb elements o tecnologies específiques. D'aquesta forma, el servidor es pot veure subjectes a canvis de forma transparent pels clients.


    \subsubsection{Codi sota petició (code on demand)}

    \paragraph{}
    Restricció que regula com, de forma excepcional, el servidor pot proporcionar accés al client sobre certes parts de la lògica del funcionament. Alguns exemples poden ser els \emph{Java Applets} o blocs de codi \emph{JavasScript}.
