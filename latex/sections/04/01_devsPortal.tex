\section{El portal de desenvolupadors}

    \paragraph{}
    Tota la informació disponible per tal de començar a familiaritzar-se amb l’\gls{API} de FamilySearch, pot ser trobada en el portal de desenvolupadors\footfullcite{fsDevelopers}.

    Aquest apartat de la web està format per diferents seccions, però, malauradament, l’estructura proposta per FamilySaerch, no acaba de resultar clara per una persona que vulgui iniciar-se per primer cop en l'ús d'aquesta \gls{API}.

    La informació es troba disgregada en molts petits apartats, situats en diferents zones del portal de desenvolupadors i sense una connexió aparent que indiqui o suggereixi un ordre de lectura. Aquest fet provoca que la informació més rellevant per comprendre l'estructura de l'API, quedi enterrada per continguts més específics i concrets, dificultant-ne la localització i absorció.

    En aquest projecte, hem desenterrat aquests continguts més bàsics i principals i els hem dotat d'un ordre de lectura, que facilita la comprensió dels fonaments tècnics sobre els quals l'API és construïda i posteriorment, exposa amb claredat, l'estructura del model de dades que conforma l'arbre familiar de FamilySearch.

    Creiem, que d'aquesta forma, s'aconsegueix proporcionar una visió global del funcionament de l'API, que permet endinsar-se en els detalls específics d'aquesta de forma molt més assequible i natural.

    Així doncs, si ens enfoquem més en la documentació disponible, que no pas en l'estructura proposada per l’organització, podem veure que la informació podria ser distribuïda en els següents grups:

    \begin{itemize}
        \item \textbf{Requisits tècnics:} Conjunt d’informació necessària per comprendre l’estructura de l'\gls{API}, els formats de dades que maneja i els passos necessaris per començar a interactuar amb aquesta.
        \item \textbf{Recursos disponibles i rutes d’accés:} Informació detallada sobre cada recurs accessible a través de l'\gls{API}. En concret, disposa dels detalls de com accedir al recurs, les operacions que es poden realitzar sobre ell, la informació que conté i quines són les connexions amb altres recursos.
        \item \textbf{Evolució i canvis produïts a l'\gls{API}:} Informació semi ordenada de com l'\gls{API} s’ha vist evolucionada al llarg del temps i un recull dels canvis produïts sobre els recursos, procés de certificació, material de documentació i eines de desenvolupament.
        \item \textbf{Serveis extres oferts per l'\gls{API}:} Aquest recull d’articles conceptualitza ca\-rac\-te\-rís\-ti\-ques de l'\gls{API} com poden ser els recursos d'\emph{emmagatzematge}, \emph{lo\-ca\-lit\-za\-ció} o \emph{throttling}.
        \item \textbf{Eines de desenvolupament:} Recull d’entorns de desenvolupament i eines extres que poden facilitar la feina dels desenvolupadors.
        \item \textbf{Certificació:} Recull la informació necessària per gestionar els diferents processos de certificació i informació sobre les regulacions a les quals s'ha de fer front, en cas de voler certificar l'aplicació.
    \end{itemize}
