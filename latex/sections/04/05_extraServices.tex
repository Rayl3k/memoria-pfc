\section{Serveis extres oferts per l'API}

    \paragraph{}
    L'API de FamilySearch no es caracteritza només per ser el centre d'informació obert amb més registres genealògics, sinó que a més a més, l'API ha estat dissenyada i pensada, per ser utilitzada de forma simultània per moltes persones, des de diferents indrets del món, d'una forma còmode i personalitzada.

    Tot i que evidentment, cap sistema és infal·lible i de fet, com ja hem comentat, FamilySearch està realitzant una actualització del seu sistema per tal de garantir-ne una millor escalabilitat, el sistema actual ja presenta una sèrie de serveis dignes de menció.

    \subsection{Caching}

        \paragraph{}
        El fet de voler oferir un servei de \emph{caching} decent, ha influenciat bona part del di\-sseny de l'API. Recordem, que la definició de \emph{caching} en el context d'una arquitectura REST, ha estat definida en l'apartat~\ref{section:42caching}.

        Davant la pregunta evident de, perquè és important el \emph{caching} a FamilySearch? Voldríem destacar els següents tres punts:

        \begin{itemize}
            \item El fet que el client pugui aplicar tècniques d'emmagatzematge, evita la repetició de certes peticions al sistema.
            \item L'existència dels \emph{reverse proxies}, permet als servidors respondre a les peticions dels clients sense necessitat d'accedir als servidors i bases de dades, si detecten que la informació demanada ja es troba disponible en aquests. Per tant, ajuden a reduir la càrrega del sistema.
            \item Davant de noves peticions, els servidors poden indicar als \emph{proxies} que les seves dades encara són vàlides i que per tant, no fa falta que realitzin una nova petició i n'esperin la resposta.
        \end{itemize}


    \subsection{Throttling}

        \paragraph{}
        La funcionalitat de \emph{throttling} s'encarrega de posar un límit en el nombre de peticions, que durant un cert interval de temps, un usuari pot realitzar. Aquests límits són creats en l'àmbit d'usuari i per tant, utilitzar més d'una sessió simultània, no permet saltar-se aquesta restricció.

        Les polítiques de \emph{throttling} permeses són calculades mitjançant el temps de processat, respecte al temps real transcorregut. Diferents recursos, disposen de diferents polítiques. En cas d'excedir aquests límits, la capçalera de resposta indicarà quant de temps caldrà esperar, per poder tornar a interactuar amb l'API.

        Aquest servei, que evidentment suposa una limitació pels usuaris, evita que certs usuaris, de forma involuntària o intencionada, bloquegin tots els recursos disponibles per part dels servidors i el servei quedi inutilitzable per altres persones.


    \subsection{Sincronització}

        \paragraph{}
        Un dels reptes més comuns per aquelles aplicacions que emmagatzemen dades d'un tercer, és el de saber si aquestes dades han estat modificades des de l'últim cop que van ser consultades o extretes.

        Per fer front a aquest repte, el paràmetre \emph{ETag} de la capçalera de les peticions contra l'API, ha estat modificat per retornar la versió del recurs consultat. Per tant, si es realitza una petició de només aquestes capçaleres, a canvi d'un petit esforç per part de l'API, es podria comparar si els recursos desitjats han estat modificats o no, sense causar un gran impacte en el sistema.

        Un cop es coneix que un recurs ha estat modificat, es poden descobrir els canvis realitzats mitjançant la comparació dels dos objectes o llegint l'historial de canvis. Evidentment, sempre es podria realitzar una sobre escriptura completa del recurs.


    \subsection{Internacionaltizació}

        \paragraph{}
        Un dels serveis que més m'ha sorprès trobar-me a l'API de FamilySearch és el del suport a la internacionalització.

        La internacionalització consisteix a adaptar una peça de software o contingut, a diferents idiomes, sense necessitat de realitzar canvis en el codi. L'API de Family\-Search permet realitzar les peticions de forma que, certs conjunts de dades re\-tor\-na\-des, com poden ser per exemple els noms de països o persones, estiguin localitzats en l'idioma especificat.

        La internacionalització permet, per exemple, que un usuari que es troba a Espanya vegi el terme `living' (viu), en el seu idioma natiu. Un altre recurs que moltes pàgines web localitzen són les dates. És sabut que l'estructura Americana d'una data, difereix d'una data europea en la posició dels termes any, mes i dia i que per tant, pot induir a errors, si aquesta no es troba localitzada de forma adequada.

        El conjunt de dades que és localitzat en l'API de FamilySearch, es llista a continuació:

        \begin{itemize}
            \item Les propietats de visualització del recurs Persona.
            \item La informació relativa a les localitzacions.
            \item Certs aspectes del vocabulari controlat o comú, de l'API.
            \item Dates.
        \end{itemize}


    \subsection{Transferència de registres en grans quantitats}

        \paragraph{}
        Existeix la possibilitat, per les organitzacions que així ho desitgin, de signar un acord amb FamilySearch que els hi permeti mantenir la seva còpia de registres genealògics, referents a persones difuntes, de les dades oficials.

        Per realitzar aquestes transferències, que mouen alts volums de dades, FamilySearch disposa d'una API especial que permet a les organitzacions gestionar, mantenir i actualitzar les dades emmagatzemades en els seus sistemes, a través de di\-fe\-rents sessions.

        Aquesta opció pot permetre a altres organitzacions ordenar i emmagatzemar les dades amb una estructura diferent i habilitar, d'aquesta forma, la possibilitat de realitzar diferents tipus d'estudi. Per exemple, estudis de mineria de dades.
