\section{Algoritme de marcatge de duplicats}

    \paragraph{}
    FamilySearch té una funció que permet, donada una persona, aconseguir les persones de l'arbre que tenen una alta probabilitat de ser un duplicat.

    L'objectiu d'aquesta aplicació seria realitzar un algoritme, que donada una persona, estudies les persones marcades com a candidates a ser un duplicat  i avalués si aquestes marques tenen pinta de ser correctes o no.

    L'estudiant podria cercar i comparar segons la diferent informació disponible de cada persona i emetre una conclusió final de diferents nivells, com per exemple:

    \begin{itemize}
        \item Insuficient informació per concloure.
        \item Duplicat descartat per inconsistència en els naixements.
        \item Duplicat real amb coincidències de dates de naixement.
        \item Falta d'informació estructurada i uniforme a causa de canvis en la informació al llarg del temps.
        \item Etcètera.
    \end{itemize}

    Un altre aspecte que el projecte podria intentar atacar és comparar la fiabilitat d'aquest algoritme amb la identificació de duplicats per part de FamilySearch.

    Aquest projecte pot resultar bastant complex i a priori, es desconeix la precisió o condicions sota les quals FamilySearch marca a una persona com a duplicada. Per tant, es recomana realitzar un bon estudi previ d'aquests conceptes abans d'embarcar-se en el projecte.
