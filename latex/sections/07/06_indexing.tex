\section{Projectes d'indexació}

    \paragraph{}
    Tot i que aquesta proposta no pretén interactuar directament amb l'API de Family\-Search, volíem realitzar com a mínim una proposta que estigués relacionada amb el procés d'indexació.

    Existeixen dos processos d'indexació diferents, els que es realitzen sobre fitxers amb un format específic i els que es basen en la transcripció d'imatges a través del software de FamilySearch. En conseqüència, aquesta proposta de projecte, pot ser dividida en dues de diferents.

    La primera, aconseguir accés a algun registre genealògic local o posar-se amb contacte amb alguna organització que vulgui pujar el contingut de registres amb un format específic, al núvol. Sobre aquests registres, s'implementaria un sistema d'automatització que transcriuria les dades i les prepararia per ser enviades a Family\-Search.

    Una organització que podría estar interessada en aquest volcatge automàtic dels registres genealògics, podría ser, per exemple, la Societat Catalana de Genealogia (SCGHSVN).

    La segona possibilitat, consisteix a realitzar un programa que interactuí amb les imatges digitalitzades per FamilySearch, pendents de ser indexades. Aquesta aplicació podria llegir les seccions de la imatge sobre les quals s'ha d'extreure la informació i intentar informar a l'usuari o omplir automàticament, el contingut dels camps.

    Aquesta idea, combina tècniques de visió per computador, sistemes d'informació i bases de dades, així que aquest projecte podria ser un candidat a ser conduït per més d'un estudiant, de forma col·laborativa.

    Aquestes dues propostes esdevenen, amb alta probabilitat, bastant complexes i per aquest motiu, es prega als estudiants que realitzin un bon estudi previ sobre l'abast i viabilitat dels objectius, abans d'inscriure el projecte.

    Una altra possibilitat d'extensió d'aquest projecte, seria coordinar-se amb Family\-Search, per tal d'intentar ampliar el procés d'indexació amb un pas opcional, que permetés exigir la validació de noms i cognoms, per una persona del mateix país que el registre a transcriure. D'aquesta forma, es podrien evitar molts errors, actualment existents, en la transcripció d'aquests camps.
