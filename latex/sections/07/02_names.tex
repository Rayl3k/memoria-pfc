\section{Comparació sobre la popularitat de noms}

    \paragraph{}
    L'objectiu d'aquesta funcionalitat és comparar la popularitat d'un o més noms, en un període concret del temps i amb la possibilitat de fixar la regió demogràfica a consultar.

    La idea principal és que donada la introducció d'un o més noms, el sistema cerqui el nombre d'instàncies de persones nascudes amb el nom especificat, en els deu anys anteriors o posteriors a la data indicada, per la regió geogràfica especificada.

    D'aquesta forma, es podria comparar quin dels dos noms ha estat més popular, segons les dades de FamilySearch, any a any.

    Al mateix temps, esdevé interessant permetre la cerca d'un sol nom per observar si certs esdeveniments històrics han pogut influenciar el nombre de nadons amb un cert nom. Per exemple, suposa l'elecció d'Obama com a president dels Estats Units, un increment en el nombre de persones nascudes amb aquest nom durant els següents anys?

    La segona raó de ser de l'eina és ajudar a decidir, per exemple, el nom dels fills d'una persona, comparant, d'aquesta forma, la popularitat actual dels noms que s'estiguin avaluant.

    A continuació llistem diferents possibilitats d'extensió:

    \begin{itemize}
        \item Ampliar la cerca a diferents països, on per cada país, el nom introduït serà localitzat. Per exemple, si l'usuari introdueix Alexander, la cerca a Espanya fos realitzada amb el nom d'Alejandro o Alex. En cas de no trobar aquesta base de dades, sempre es podria crear una taula manual d'exemple, amb unes quantes llengües i noms i utilitzar-la.
        \item Comparar instàncies de noms a Catalunya extrets de les dades de FamilySearch, amb la comparació real extreta del institut nacional d'estadística.
    \end{itemize}
