\section{Estudis estadístics sobre els noms i cognoms}

    \paragraph{}
    Aquesta proposta de projecte pretén estudiar certs comportaments coneguts sobre l'ús dels noms i cognoms, en diferents indrets geogràfics, a través de FamilySearch.

    El primer anàlisi que suggerim és realitzar una comparació sobre quants noms de pila acostumen a portar les persones, segons la seva localització geogràfica i moment en el temps.

    Per exemple, es podria comparar la mitjana de noms de pila pels principals països de cada continent i observar, si els resultats obtinguts, s'ajusten a la realitat que ens esperàvem i per què.

    Un segon anàlisi estadístic que pot ser conduït, gira al voltant dels cognoms de les dones. Durant molts anys i en diferents indrets geogràfics, les dones, al néixer, eren referenciades pel seu cognom patern i posteriorment, pel cognom del marit un cop es casaven.

    L'anàlisi que es proposa des d'aquí és estudiar quants cognoms diferents han estat atribuïts al col·lectiu de les dones en concret i com aquest valor ha anat evolucionant al llarg del temps.

    Com a possibilitat extra, també es pot intentar realitzar un estudi sobre en quina edat mitjana, per cada època, s'acostumaven a produir aquests canvis de cognom per matrimoni.
