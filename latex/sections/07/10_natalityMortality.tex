\section{FamilySearch i la segona guerra mundial: Natalitat i Defuncions}

    \paragraph{}
    Durant el transcurs del temps han succeït un gran nombre d'esdeveniments que han afectat la població mundial de diferents formes. Un dels conflictes que ha causat més repercussió, ha estat la segona guerra mundial, encara que qualsevol guerra local o invasió de territoris, ha deixat una impremta pel que fa al nombre d'enterraments en les poblacions afectades o matrimonis que succeeixen durant les invasions i ja es queden a residir en aquella població.

    L'objectiu d'aquesta proposta de projecte és observar l'impacte que va tenir aquest esdeveniment, a través dels índexs de natalitat i defunció i l'aparició de nous cognoms, clarament estrangers, en els països implicats en el conflicte. Es recomana ampliar la finestra de temps estudiat més enllà dels anys del conflicte, per observar quins eren els valors normals, previs i posteriors, a la segona guerra mundial.

    Per aquesta proposta de projecte, l'estudiant haurà de trobar la forma d'escalar les dades de cada país segons el volum de registres disponibles, en cas de voler comparar diferents països de forma directa.

    Una altra tasca que podrien realitzar els estudiants és comparar els valors obtinguts a través de FamilySearch, amb les dades oficials de la segona guerra mundial i respondre preguntes de l'estil: Es corresponen els països amb un increment de defuncions més elevat, amb els que van patir més baixes durant la segona guerra mundial?
