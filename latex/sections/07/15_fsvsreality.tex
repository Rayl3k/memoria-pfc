\section{Comparació de dades genealògiques reals amb Family\-Search}

    \paragraph{}
    Aquesta proposta de projecte pretén validar, en certa forma, com de bé representen les dades de FamilySearch la realitat d'un país o països, a través de diferents èpoques o per una època determinada.

    Aquesta proposta podria ser dividida en molts projectes diferents, un per cada país, concepte o època que l'estudiant vulgui explorar. Per ajudar a comprendre als estudiants al que ens estem referint, a continuació citem una sèrie d'exemples:

    \begin{itemize}
        \item Comparació de mortalitat infantil: Global, per continents, països, èpoques, etcètera.
        \item Mitja de fills per família: Global, per continents, països, èpoques, etcètera.
        \item Comparació de la densitat d'habitants en diferents països, estats, etcètera.
        \item Edat mitjana de les persones donat un moment i localització concreta. Tam\-bé és podria mirar l'evolució d'aquest indicador al llarg del temps.
        \item Proporció d'homes i dones: Global, per continents, països, èpoques, etcètera.
        \item Esperança de vida per les dones i homes: Global, per continents, països, èpoques, etcètera.
    \end{itemize}

    Aquesta proposta no pretén que l'estudiant abordi tots els conceptes diferents que es pugui imaginar, però si aquells que cregui que poden tenir un valor més elevat de cara a comparar regions i èpoques.

    Segons els factors a estudiar, la complexitat de com hauran de ser processades les dades canvia i per tant, caldrà tenir-ho en compte a l'hora de definir l'abast del projecte.

    Es recomana també als estudiants que cerquin estudis estadístics sobre la població mundial, per poder inspirar-se de cara al plantejament de propostes de projecte.

    A nosaltres, per exemple, ens van ajudar molt en al creació d'aquesta proposta de projecte, la presentació a les conferències Ted de Kim Preshoff\footfullcite{populationPyramids} o les principals característiques de la població mundial al 2014\footfullcite{populationHighlights}.
