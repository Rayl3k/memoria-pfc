\section{Projectes d'indexació}

    \paragraph{}
    La secció d’indexació posa a disposició dels usuaris tota la informació necessària per convertir-se en voluntaris i començar a transcriure informació continguda en registres genealògics digitalitzats.

    El procés d’indexació es realitza a través d’un programa creat per FamilySearch que pot ser descarregat des de la mateixa web i s’encarrega de gestionar els registres que l’usuari pot indexar segons els projectes en els quals aquest es trobi inscrit.

    Al mateix temps, FamilySearch està a punt de treure un nou servei que permetrà la indexació en línia des del mateix navegador.


    \subsubsection{Procés d'indexació}

        \paragraph{}
        El cicle de vida d’un registre consisteix en les següents fases:

        \begin{enumerate}
            \item Digitalització del registre: En aquesta fase els arxius i registres es digitalitzen.
            \item Agrupació dels registres en grups de 20--50 camps diferents dels que cal extreure informació.
            \item Dos voluntaris diferents transcriuen els valors dels camps digitalitzats.
            \item Si la informació introduïda pels dos voluntaris no coincideix, un àrbitre avalua les dues entrades i pren una decisió.
            \item Si les dues extraccions coincideixen, o un cop han estat validades per un àrbitre, les dades s’envien a una base de dades i són preparades per la seva publicació.
        \end{enumerate}

        \paragraph{}
        Transcriure dades, sobretot aquelles que provenen de documents antics, no és simple i en conseqüència, el programa d’indexació permet introduir com a resposta que part dels camps del registre no poden ser transcrits o que no s'entén el que posa.

        De la mateixa forma, FamilySearch proporciona exemples d’escriptura antiga, en diferents idiomes, per ajudar als usuaris a desxifrar els camps més complicats.


    \subsubsection{Què pot indexar un voluntari?}

        \paragraph{}
        Els voluntaris poden escollir un o varis dels diferents projectes que actualment es troben en el procés d’indexació i participar en ells. L’usuari pot escollir amb completa llibertat si vol treballar en projectes d’un país o llengua específica i la quantitat de temps que vol dedicar-hi.
