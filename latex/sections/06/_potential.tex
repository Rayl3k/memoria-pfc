\chapter{Valoració final sobre la potencialitat de l'API}

\section{Introducció}

    \paragraph{}
    Aquesta secció de la memòria pretén cobrir, des del meu punt de vista personal basat tant en el coneixement adquirit mitjançant l'estudi teòric de l'API, com en les petites pinzellades tècniques que s'han pogut aprendre durant la implementació dels exemples, el potencial que emmascara aquesta API de cara a generar propostes de projecte per futurs estudiants.

    En conseqüència, a pesar de la localització dins de la memòria d'aquest apartat, aquest és redactat després d'adquirir el coneixement pràctic bàsic, gràcies a la implementació dels exemples, que seran exposats en les següents seccions de la memòria.

    Per comprendre la potencialitat global d'aquesta API, cal dividir-ne l'estudi en diferents blocs o peces. Intentar emetre un judici de valor global, sense sintetitzar-ne primer diferents oportunitats i complicacions de conceptes més específics, no aconseguiria transmetre la profunditat i complexitat de l'abast d'aquest pregunta.

\section{Distribució geogràfica de les dades}

    \paragraph{}
    Per comprendre la potencialitat global d'aquesta API, cal dividir-ne l'estudi en diferents blocs o peces. Intentar emetre un judici de valor global, sense sintetitzar-ne primer diferents oportunitats i complicacions de conceptes més específics, no aconseguiria transmetre la profunditat i complexitat de l'abast d'aquest pregunta.

    FamilySearch, pot presumir de tenir una de les bases de dades d'informació genealògica oberta al públic més gran del món, si no la més gran, amb més de quatre bilions de registres.

    A pesar que el nombre de 4 bilions pugui semblar molt elevat, si el comparem amb els més de 33 bilions de persones que han nascut aproximadament des del 1200 fins a l'any 2011 (agafant així, una dada de referència pública que s'ajusti més o menys al període de temps sobre el que FamilySearch disposa d'informació), ens adonem del fet que disposem d'una mostra acceptable, però lluny de suposar una representació real.

    Cal també tenir en compte que els 4 bilions de registres emmagatzemats a FamilySearch no es troben repartits de forma proporcional sobre les diferents regions o països, sinó que l'organització, de forma evident, disposa més dades en aquells indrets en què històricament ha tingut més presència o facilitat d'accés a dades.

     D'aquesta forma, els Estats Units d'Amèrica, amb 991 col·leccions de dades diferents, ofereix la informació de 2,5 bilions de registres que daten entre els anys 1500 i 2015. En altres paraules, un 62\% del volum total de dades.

     De forma paral·lela, i per oferir una escala diferent, Espanya, amb 45 col·leccions, ofereix la informació de 24 milions de registres compresos entre els anys 1251 i 2013. Per tant, resulta fàcil observar que la dispersió de les dades i volum difereix molt segons la regió que vol ser consultada.

     Cal doncs, tenir en compte aquesta limitació de cara a proposar o realitzar certa mena de projectes, com poden ser per exemple, els estudis estadístics d'aspectes demogràfics. Com a recomanació, s'encoratja als estudiants a utilitzar aquelles regions, com els Estats Units, més plegades de registres, de cara a funcionalitats generals.


 \section{Dades contamporànies}

    \paragraph{}
    Un dels inconvenients de les dades genealògiques és que aquestes generalment es troben subjectes a lleis de protecció, durant períodes de temps prolongats, abans de poder fer-se públiques. És més, en casos especials com els que hem esmentat en les primeres seccions de la memòria, aquestes poden inclús no arribar mai al domini públic.

    Aquest aspecte implica que el valor percebut, del conjunt de dades disponible a través de FamilySearch, sigui més elevat en projectes enfocats al passat, que no pas estudis més contemporanis.

    La ironia en aquest punt de la memòria és que a menys que les legislacions canviïn  per complet, l'afirmació de què sempre es disposarà de menys dades contemporànies, seguirà sent aplicable independentment dels anys que passin.

    Un altre fet que pot impactar a la quantitat de registres contemporanis disponibles és la quantitat d'afiliacions a l'Església de Jesucrist dels Sants dels Darrers Dies i és que cal no oblidar, que aquesta organització, representa al principal benefactor de FamilySearch i per tant, principal origen de font de dades.


\section{Recursos i funcionalitats}

    \paragraph{}
    El conjunt de recursos utilitzables i les funcionalitats creades al seu voltant, esdevenen un dels punts més favorables de l'API.

    El conjunt de paràmetres accessibles relacionats a una persona, o qualsevol altre recurs, resulta immens. Des dels esdeveniments principals relacionats a la seva vida d'una persona, fins a petits detalls com els diferents noms que la persona va rebre al llarg de la seva vida. La informació es troba molt ben estructurada i tots elements relacionats, resulten fàcilment accessibles.

    La robustesa de les dades tampoc és cap broma, mitjançant el sistema de canvis es pot desfer qualsevol ús malintencionat o involuntari sobre el conjunt de dades. Clarament, FamilySearch es pren molt seriosament poder garantir la qualitat de les dades.

    Per si tot el conjunt de recursos accessible no fos suficient, FamilySearch posa a la disposició dels usuaris una sèrie de funcions de conveniència que permeten, entre altres exemples, cercar persones duplicades, accedir a les ascendències i descendències d'una persona de forma reglada i estructurada i delimitar les cerques per més paràmetres dels que un es podria imaginar.

    Tot plegat, el conjunt de recursos, granularitat de la informació i les funcionalitats de fàcil accés, converteixen l'API de FamilySearch en un poderós aliat de cara a la recerca genealògica.


\section{Naturalesa de l'API}

    \paragraph{}
    Un dels primers xocs que em vaig emportar quan vaig començar a estudiar més a fons la potencialitat de l'API, va ser perquè fins aquell moment no havia tingut en compte el motiu pel qual aquesta havia estat concebuda.

    L'API de FamilySearch neix per ajudar a individus particulars a realitzar recerca genealògica, sobre els seus avantpassats o els d'un tercer. Cal tenir molt present aquesta definició, ja que limita o debilita en gran mesura, els possibles projectes a realitzar.

    FamilySearch va dissenyar l'API perquè un usuari pogués realitzar una cerca, de la forma més específica possible, després de la recopilació prèvia d'informació sobre la persona cercada. Per tant, no està pensada per accedir a un gran volum de registres de forma simultània, accedir a les dades per un nivell de granularitat inferior a la del concepte `persona', ni realitzar moltes peticions consecutives contra la plataforma.

    En conseqüència tota aspiració de realitzar projectes de mineria de dades o estudis de baixa granularitat, queden completament descartats, o si més no, condicionats en gran mesura de cara a la implementació tècnica o automatització de tasques.


\section{Utilització de l'API en el marc d'un PFC}

    \paragraph{}
    Un últim concepte a destacar és com encaixa la utilització d'aquesta API en el marc d'un projecte final de carrera. És a dir, si deixem de banda la discussió de com és de potent aquesta, resulta factible utilitzar-la de cara a un projecte final de carrera?

    Crec que resulta de vital importància respondre a aquesta pregunta de forma independent a la de la potencialitat, per no vincular dos conceptes diferents.

    Un projecte final de carrera es desenvolupa generalment en el període de temps equivalent al d'un quadrimestre. Com ja s'ha esmentat en altres seccions de la memòria i encara tornarà a aparèixer més endavant, per tal d'aconseguir accés a les dades de producció de FamilySearch, cal certificar l'aplicació.

    Aquest procés, com aquest projecte n'és una clara mostra, pot resultar complicat i ple de complicacions, i encara podria esdevenir més complex si es volgués realitzar una aplicació amb drets d'escriptura a l'arbre genealògic o comercialitzable.

    El que volem indicar en aquest apartat és que si la planificació del projecte no és bona, i inclús així, s'incorre en un cert risc, existeix la possibilitat de no disposar del temps suficient per implementar, certificar i extreure les conclusions necessàries, sobre les dades de producció.

    El fet que existeixi en l'actualitat, la possibilitat de certificar les aplicacions per ús personal i no només comercial, augmenta en gran mesura les possibilitats dels estudiants a aventurar-se en projectes de recerca genealògica.

    En certa forma, esdevé probable que la utilització d'aquesta API condicioni l'estructura dels projectes de la mateixa forma que ho ha fet amb aquest. Forçant un clar esforç inicial per acabar la implementació tan aviat com es pugui, per tal de poder experimentar amb les dades de producció, o en el nostre cas, esbrinar de què es tractava exactament aquest procés de certificació.

    Això no obstant, s'espera que l'estudi realitzat en aquest projecte representi una facilitació i acceleració considerable de la corba d'aprenentatge pels futurs estudiants, el que els permetria accedir abans a producció.

    En resum, si, l'API de FamilySearch és un recurs utilitzable de cara a la realització de projectes finals de carrera, però cal tenir en compte les seves peculiaritats de cara a la planificació i pot resultar més atractiu a aquelles persones que pretenguin tenir clar, el projecte que volen realitzar en profunditat, abans de matricular-lo o inclús començar-lo amb un quadrimestre d'antelació, al que serà matriculat.


\section{Conclusió}

    \paragraph{}
    Considerades les limitacions geogràfiques, temporals, estructurals i temporals (en l'àmbit de temps del que es disposa per realitzar un projecte) podria semblar que no queden moltes opcions possibles, de cara a plantejar propostes de projecte, més enllà de la recerca genealògica bàsica. Malgrat això, no és la meva opinió  exacta que aquest en sigui el cas.

    Si bé és cert, que cal pensar en propostes de projecte que encaixin dins del marc delimitat per aquestes restriccions, les eines posades a disposició dels usuaris i la quantitat d'informació disponible, permeten la utilització de vies secundàries per tal d'assolir diferents objectius i ajudar així a respondre certes preguntes que aquest projecte no ha tingut temps d'explorar.

    En la següent secció de la memòria es podran observar un conjunt de propostes de projecte que encaixen dins del marc descrit en aquesta secció i que pretenen oferir resposta, a certes preguntes més específiques, sobre la potencialitat o possibilitats d'ús d'aquesta.
