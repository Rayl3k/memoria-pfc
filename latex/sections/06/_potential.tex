\chapter{Valoració final sobre la potencialitat de l'API}

\section{Introducció}

    \paragraph{}
    Aquesta secció de la memòria pretén cobrir, des del meu punt de vista personal, basat tant en el coneixement adquirit mitjançant l'estudi teòric de l'API com en les petites pinzellades tècniques que s'han pogut aprendre durant la implementació dels exemples, el potencial que emmascara aquesta API de cara a generar propostes de projecte per futurs estudiants.

    En conseqüència, a pesar de la localització dins de la memòria d'aquest apartat, aquest és redactat després d'adquirir el coneixement pràctic bàsic, gràcies a la implementació dels exemples, que seran exposats en futures seccions de la memòria.

    Per comprendre la potencialitat global d'aquesta API, cal dividir-ne l'estudi en diferents blocs o peces. Intentar emetre un judici de valor global, sense sintetitzar-ne primer les diferents oportunitats i complicacions de conceptes més específics, no aconseguiria transmetre la profunditat i complexitat de l'abast d'aquest pregunta.

\section{Distribució geogràfica de les dades}

    \paragraph{}
    L'element principal de les bases de dades de FamilySearch són els registres. És a dir, un conjunt de documents genealògics que contenen informació sobre naixements, defuncions, casaments, registres d'emigració i molts altres conceptes, de diferents persones.

    De fet, FamilySearch pot presumir de tenir una de les bases de dades d'informació genealògica oberta al públic, més gran del món, si no la més gran, amb més de quatre bilions de registres.

    A pesar que el nombre de 4 bilions pugui semblar molt elevat, si aquest és comparat amb els més de 33 bilions de persones que han nascut aproximadament des del 1200 fins a l'any 2011\footfullcite{worldPopulation} (agafant així, una dada de referència pública que s'ajusti més o menys, al període de temps sobre el que FamilySearch disposa d'informació), ens adonem del fet que disposem d'una mostra de registres acceptable, però molt lluny de suposar una representació real.

    Cal també tenir en compte que els 4 bilions de registres emmagatzemats a Family\-Search no representen 4 bilions de persones diferents, sinó que d’una mateixa persona, es poden contenir diferents registres. Per exemple, registres sobre la data de naixement, defunció, etcètera.

    A més a més, aquests registres no es troben repartits de forma proporcional sobre les diferents regions o països, sinó que l'organització, de forma evident, disposa de més dades per aquells indrets en els quals històricament ha tingut més presència o facilitat d'accés a les dades.

     D'aquesta forma, els Estats Units d'Amèrica, amb un total de 991 col·leccions de dades diferents, ofereix la informació de 2.5 bilions de registres, que daten entre els anys 1500 i 2015. En altres paraules, un 62\% del volum total de dades.

     De forma paral·lela i per oferir una escala diferent, Espanya, amb 45 col·leccions, ofereix la informació de 24 milions de registres, compresos entre els anys 1251 i 2013. Per tant, resulta fàcil observar que la dispersió de les dades i volum difereix molt segons la regió que vol ser consultada.

     Cal doncs tenir en compte aquesta limitació de cara a proposar o realitzar certa mena de projectes, com poden ser per exemple, els estudis estadístics d'aspectes demogràfics.

     Com a recomanació, s'encoratja als estudiants a utilitzar aquelles regions com els Estats Units, més plegades de registres, de cara a funcionalitats generals o centrar-se en dades del seu país (per exemple, Espanya), per proporcionar resultats de més interès local. Això si, tenint en compte les possibles limitacions existents depenent del nombre de registres genealògics disponibles.


 \section{Dades contemporànies}

    \paragraph{}
    Un dels inconvenients de les dades genealògiques és que aquestes generalment es troben subjectes a lleis de protecció, durant períodes de temps prolongats, abans de poder fer-se públiques. És més, en casos especials, com els que hem esmentat en les primeres seccions de la memòria, aquestes poden inclús no arribar mai al domini públic.

    De fet, a Espanya, els registres genealògics no poden considerar-se públics fins cent anys després de la data de l'esdeveniment enregistrat o en cas de conèixer la data de defunció de la persona, fins 50 anys després de la seva mort.

    Aquest fet implica, que de forma general, FamilySearch no pugui contenir re\-gis\-tres genealògics provinents de fonts de dades públiques Espanyoles, posteriors al 1916, en el moment de redactar aquesta memòria. Tota informació disponible posterior a aquesta data, ha d'haver estat adquirida a través de particulars.

    Tot i que cada país pot tenir les seves pròpies lleis, respecte al temps que ha de transcórrer abans de considerar com a públiques les dades de caràcter genealògic, el varem de 100 anys és bastant acceptat, en major o menor mesura, de forma generalitzada.

    Aquest aspecte implica que el valor percebut del conjunt de dades disponible a través de FamilySearch, sigui més elevat en projectes enfocats en el passat, que no pas en estudis contemporanis.

    La ironia d'aquest punt de la memòria és que a menys que les legislacions canviïn per complet, l'afirmació de què sempre es disposarà de menys dades contemporànies, seguirà sent aplicable independentment dels anys que transcorrin.

    Un altre fet que pot impactar a la quantitat de registres contemporanis disponibles és la quantitat d'afiliacions a l'Església de Jesucrist dels Sants dels Darrers Dies i és que cal no oblidar, que aquesta organització, representa al principal benefactor de FamilySearch i per tant, principal origen de font de dades.


\section{Recursos i funcionalitats}

    \paragraph{}
    El conjunt de recursos utilitzables i les funcionalitats creades al seu voltant, esdevenen un dels punts més favorables de l'API.

    El conjunt de paràmetres accessibles relacionats a una persona o qualsevol altre recurs, resulta immens, des dels esdeveniments principals relacionats a la vida d'una persona, fins a petits detalls com els diferents noms que la persona va rebre al llarg de la seva vida. La informació es troba molt ben estructurada i tots els elements relacionats, resulten fàcilment accessibles.

    La robustesa de les dades tampoc ha de ser pressa a la lleugera, ja que el sistema de canvis implementat pot desfer qualsevol acció malintencionada o involuntària, que malmeti el conjunt de registres genealògics. Clarament, FamilySearch es pren molt seriosament poder garantir la qualitat i veracitat de les seves dades.

    A més a més, per si tot el conjunt de recursos accessible no fos suficient, Family\-Search posa a la disposició dels usuaris una sèrie de funcions de conveniència que permeten, entre altres exemples, cercar persones duplicades, accedir a les ascendències i descendències d'una persona de forma estructurada i reglada i delimitar les cerques per més paràmetres dels que un es podria imaginar.

    Tot plegat, el conjunt de recursos disponibles, la granularitat de la informació i les funcionalitats de fàcil accés, converteixen l'API de FamilySearch en un poderós aliat de cara a la recerca genealògica.


\section{Naturalesa de l'API}

    \paragraph{}
    Un dels primers xocs que em vaig emportar quan vaig començar a estudiar més a fons la potencialitat de l'API, va ser el no haver tingut en compte el motiu pel qual aquesta havia estat concebuda.

    L'API de FamilySearch neix per ajudar a individus particulars a realitzar recerques genealògiques sobre els seus avantpassats o els d'un tercer. Cal tenir molt present aquesta definició, ja que limita o debilita en gran mesura, les possibilitats d'explotació de les dades contingudes per l'API.

    FamilySearch va dissenyar l'API perquè un usuari pogués realitzar una cerca, de la forma més específica possible, després d'haver recopilat tota la informació ja coneguda de la persona cercada. Per tant, no està pensada per accedir a un gran volum de registres de forma simultània, accedir a les dades per un nivell de granularitat inferior a la del concepte `persona', ni realitzar moltes peticions consecutives contra la plataforma.

    En conseqüència, tota aspiració de realitzar projectes de mineria de dades o estudis de baixa granularitat, queden completament descartats o si més no, condicionats en gran mesura, de cara a la implementació tècnica o automatització de les tasques necessàries.


\section{Utilització de l'API en el marc d'un PFC}

    \paragraph{}
    Un últim concepte a destacar és com encaixa la utilització d'aquesta API en el marc d'un projecte final de carrera. És a dir, si deixem de banda la discussió de com potent és aquesta, resulta factible utilitzar-la de cara a un projecte final de carrera?

    Crec que resulta de vital importància respondre a aquesta pregunta de forma independent a la de la potencialitat, per no vincular dos conceptes diferents.

    Un projecte final de carrera es desenvolupa generalment en el període de temps equivalent al d'un quadrimestre. Com ja s'ha esmentat en altres seccions de la memòria i encara tornarà a aparèixer més endavant, per tal d'aconseguir accés a les dades de producció de FamilySearch, l'aplicació ha de ser certificada per l'organització.

    Aquest procés, com aquest projecte n'és una clara mostra, pot resultar complicat i encara podria esdevenir més complex, si es volgués realitzar una aplicació commercialitzable o amb drets d'escriptura a l'arbre genealògic. Cal també mencionar, que és probable que aquest projecte s'hagi vist afectat per circumstàncies extraordinàries (problemes en la web oficial), exponenciades a causa d'haver iniciat els contactes en ple estiu.

    El que volem transmetre en aquest apartat és que si la planificació del projecte no és bona, existeix la possibilitat de no disposar del temps suficient per implementar, certificar i extreure les conclusions necessàries, sobre les dades de producció. Doncs alguns dels problemes sobre el tractament de les dades, no podran ser descoberts a través de l'entorn de proves i no serà fins a arribar a producció, que aquests podràn ser detectats i tractats.

    De totes maneres, el fet que existeixi en l'actualitat la possibilitat de certificar les aplicacions per un ús personal i limitat (només els desenvolupadors i beta testers), facilita en gran mesura l'obtenció d'un accés ràpid a producció, pel que només fa falta disposar d'un prototip funcional. Per tant, crec que els estudiants no s'han de preocupar excessivament per aquest aspecte.

    Això sí, és probable que la utilització d'aquesta API condicioni l'estructura dels projectes de la mateixa forma que ho ha fet amb aquest, forçant un clar esforç inicial per obtenir un prototip funcional que pugui experimentar amb les dades de producció, o en el nostre cas, una aplicació que ens permetés esbrinar en què consistia exactament aquest procés de certificació.

    En resum, si, l'API de FamilySearch és un recurs utilitzable de cara a la realització de projectes finals de carrera, però cal tenir present aquest procés de certificació de cara a la planificació. Es recomana als estudiants que vulguin utilitzar aquesta API, que tinguin una idea inicial clara del que volen estudiar a esbrinar, per garantir així una utilització eficient del temps des del mateix moment d'inscripció del projecte.


\section{Conclusió}

    \paragraph{}
    Considerades les limitacions geogràfiques, estructurals i temporals (en l'àmbit de temps del que es disposa per realitzar un projecte), podria semblar que no queden moltes opcions possibles de cara a plantejar propostes de projecte, més enllà de la recerca genealògica bàsica. Malgrat això, no és la meva opinió exacta que aquest en sigui el cas.

    Si bé és cert, que cal pensar en propostes de projecte que encaixin dins del marc delimitat per aquestes restriccions, les eines posades a disposició dels usuaris i la quantitat d'informació disponible, ampliada a través d'aquest projecte, permeten la utilització de vies secundàries per tal d'assolir diferents objectius i ajudar durant la realització de projectes relacionats amb l'API de FamilySearch.

    Al mateix temps, el procés de certificació no és un camí tan obscur com podia semblar, principalment, gràcies a l'existència d'un accés limitat a producció per prototips funcionals i si els contactes amb l'organització es gestionen amb un marge de temps suficient, no hi hauria d'haver cap problema.

    Així doncs, en la següent secció de la memòria, es podran observar un conjunt de propostes de projecte que encaixen dins del marc descrit en aquesta secció i que pretenen oferir resposta a certes preguntes més específiques, sobre la potencialitat o possibilitats d'ús, d'aquesta API.
