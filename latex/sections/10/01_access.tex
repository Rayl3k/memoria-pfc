\section{Accés a l'aplicació web i codi de l'aplicació}

    \paragraph{}
    L’aplicació web es troba desplegada al núvol sota l'URL:

    \begin{displayquote}
        https://pfc-fs-api-potential.herokuapp.com/
    \end{displayquote}

    Per accedir a la zona específica d’exemples, la que s’encarrega de mostrar els diferents exemples d’interacció amb l’API, fa falta utilitzar un compte d’usuari de FamilySearch. En l’annex A, han estat incloses tres comptes de prova pels membres del tribunal, però en cas de preferir-ho, cada persona interessada pot crear-se un compte d’usuari propi a FamilySearch\footnote{http://familysearch.org/register/}.

    Per altra banda, el codi de les aplicacions web sol ser extens en nombre de línies i mostrar-lo en detall en aquesta memòria resulta impracticable. El codi desenvolupat ocupa un total de [nombre de línies] repartides en un  [x\%] en HTML, un [y\%] en Javascript i jQuery i un [z\%] en css.

    Tot el codi de l’aplicació pot ser trobat en el repositori GitHub accessible a través del següent URL:

    \begin{displayquote}
        https://github.com/sinh15/pfc-family-search
    \end{displayquote}

    L’estructura del codi serà presentada més endavant en aquesta mateixa secció de la memòria, però principalment, el servidor està compost pel fitxer \emph{app.js}, els fitxers HTML poden ser trobats a la carpeta \emph{views} i els fitxers Javascript i jQuery són accessibles a través de la carpeta \emph{assets}.
