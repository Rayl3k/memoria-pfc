\subsection{Identificació amb FamilySearch}

    \paragraph{}
    Aquesta pàgina s'utilitza per assegurar que l'usuari no pot utilitzar els exemples implementats, sense identificar-se primer a l’API de FamilySearch.

    La pàgina apareix quan l'usuari intenta accedir a la pàgina d'exemples o la pàgina d'un exemple en concret, però encara no s’ha identificat amb FamilySearch. La pàgina permet dues accions simples, tornar enredera (o a la home si s'ha accedit a la pàgina mitjançant la introducció directa de l'URL) o identificar-se amb FamilySearch.

    El procés d'identificació, s'inicia mitjançant el llançament d'un pop-up, que obre la pàgina de FamilySearch. Aquesta, demana la introducció del nom d'usuari i contrasenya. Un cop aquesta informació ha estat verificada, el servidor redirigeix a l'usuari a la pàgina que havia demanat accedir.

    La pàgina d'identificació, tampoc pateix cap reestructuració especial del contingut quan es veu amb dispositius més petits. Simplement, s'adapta a la pantalla que la mostra i reorganitza els botons que permeten tornar endarrere o obrir el procés d’identificació.
