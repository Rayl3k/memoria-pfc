\section{Fitxers de l'aplicació i la seva funcionalitat}

    \paragraph{}
    En aquest apartat de la memòria, volem presentar l'arbre d'arxius generats, resultant de programar l'aplicació web i exposar-ne, de forma bàsica, la funció que desenvolupa cada un d'ells en el marc de l'aplicació.

    Recordem, que tot el codi programat per tal de fer funcionar l’aplicació web, pot ser trobat a l'URL:

    \begin{displayquote}
        https://github.com/sinh15/pfc-family-search
    \end{displayquote}

    Com comentavem als inicis d'aquesta secció de la memòria, el conjunt de fitxers programats representa un total de 5.731 línies de codi, de les quals, un 35.88\% són codi HTML, un 55.89\% codi Javascript i un 8.24\% codi CSS.

    La taula~\ref{tab:codeFiles} mostra la localització relativa de cada arxiu i en descriu breument la seva funcionalitat.

    \begin{center}
             \csvreader[
                separator=semicolon,
                before table=\sffamily\small,
                longtable={p{4cm-2\tabcolsep}p{10cm-2\tabcolsep}},
                table head={\caption{Fitxers de codi de l'aplicació web}\label{tab:codeFiles}\\\toprule%
                    \headentry{m{4cm-2\tabcolsep}}{Arxiu}
                    & \headentry{m{10cm-2\tabcolsep}}{Funció}\\\midrule},
                late after line=\\\midrule,
                late after last line=\\\bottomrule,
             ]
             {./tables/10/codeFiles.csv}
             {name=\name,desc=\desc}
             {\name&\desc}
     \end{center}
