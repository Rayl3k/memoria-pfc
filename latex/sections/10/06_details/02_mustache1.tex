\subsection{Estructura general d'una pàgina HTML: Mustache I}

    \paragraph{}
    A excepció de la pàgina d'identificació amb FamilySearch, la resta de pàgines de l'aplicació web segueixen la mateixa estructura, tal com s'ha exposat en l'apartat~\ref{sec:introStructure}.

    Per evitar la duplicació de codi entre les diferents pàgines, els elements compartits han estat creats en fitxers separats i són carregats a cada pàgina segons si volem mostrar-los o no. Recordem que la tecnologia utilitzada, que fa això possible, és el llenguatge de plantilles Mustache.

    L'esquelet del codi d'una pàgina qualsevol del nostre web segueix la forma mos\-tra\-da en el bloc de codi següent.

    \begin{lstlisting}[style=rawOwn,caption={Exemple d'inclusió de fitxers HTML amb Mustache}]
<!-- header includes -->
{{> header }}
{{> navbar }}
{{> pageTitle }}
<!-- specific page content -->
...
{{> javascripts }}
<!-- specific javascripts for this file -->
...
<!-- footer -->
{{> footer }}
    \end{lstlisting}

    Les etiquetes \emph{\{\{> fileName\}\}}, s'utilitzen per indicar que es vol importar, en aquesta posició, el codi HTML del fitxer \emph{fileName.html}.

    Mitjançant aquesta estructura tenim el control absolut de quins components volem incloure a cada una de les pàgines. Per exemple, l'esquelet de la pàgina d'identificació, es diferencia de la resta en el fet que no s'inclouen els fitxers \emph{navbar}, \emph{pageTitle} ni  \emph{footer}.

    També cal destacar que abans del \emph{footer} deixem un espai en blanc per incloure fitxers Javascript, que són utilitzats únicament per la pàgina carregada. D'aquesta forma, s'evita carregar tots els controladors a totes les pàgines, si aquests no han de ser utilitzats.
