\subsection{El fitxer index.html: El grid de Bootstrap I}\label{sec:bootstrap1}

    \paragraph{}
    Aprofitarem aquest fitxer per explicar en profunditat el funcionament del grid de Bootstrap, introduït ja en la secció~\ref{sec:bootstrap} de la memòria.

    El contingut principal del fitxer index.html, representa un conjunt d'enllaços als tres grans blocs de l'aplicació web, on cada un, es veu representat per una imatge, un títol i una descripció. En aplicacions d'escriptori, aquests tres blocs es mostren un al costat de l'altre, mentre que en aplicacions de pantalles més reduïdes, es mostren un sobre l'altre.

    Per aconseguir-ho, es juga amb el concepte de files i columnes del grid de Boots\-trap. El grid, està format per contenidors o blocs d'espai, en els que es poden crear diferents fileres i cada una d'aquestes fileres, pot ser dividida en dotze columnes.

    Com que en la versió d'escriptori volem crear tres blocs idèntics, on cada un contindrà un enllaç a una secció de la pàgina web, a cada un dels blocs li corresponen 12/3 = 4 columnes. Per indicar-ho a bootstrap, s'utilitza la classe, \emph{col-md-4}.

    Així doncs, el codi extremadament simplificat d'aquesta secció, podria ser representat de la següent forma:

    \begin{lstlisting}[style=rawOwn,caption={Exemple bàsic de divisió d'una filera en tres columnes}]
<div class=`container'>
    <div class=`row'>
        <div class=`col-md-4'> ... </div>
        <div class=`col-md-4'> ... </div>
        <div class=`col-md-4'> ... </div>
	</div>
</div>
    \end{lstlisting}

    La configuració mostrada en el bloc de codi anterior representaria exactament el format que volem que la pàgina agafi per escriptoris, però perquè aquesta estructura funciona també en els dispositius mòbils, apilant-ne els blocs de quatre columnes un sobre l'altre?

    Per comprendre-ho, hem d'explicar primer, que Bootstrap, categoritza les pantalles en quatre grandàries diferents, segons l'amplada en píxels del dispositiu.

    \begin{itemize}
        \item \textbf{Dispositius extra reduïts (xs):} Fa referència als dispositius mòbils o pantalles amb amplitud inferior als 768 píxels.
        \item \textbf{Dispositius petits (sm):} Fa referència sobretot a tauletes gràfiques o pantalles amb amplitud superior o igual a 768 píxels, però inferior a 992 píxels.
        \item \textbf{Dispositius mitjans (md):} Fa referència a escriptoris o pantalles amb amplitud superior o igual als 992 píxels i inferior als 1200 píxels.
        \item \textbf{Dispositius grans (lg):} Fa referència a escriptoris grans o pantalles amb amplitud superior o igual als 1200 píxels.
    \end{itemize}

    Les lletres que s'han indicat entre parestèsies en el llistat anterior, són un codi que s'introdueix en la classe columna de Bootstrap (\emph{col-XX-4}), per indicar sobre quina grandària mínima, s'ha d'aplicar la regla.

    D'aquesta forma, la classe utilitzada en el bloc de codi d'exemple (\emph{col-md-4}), indica que es vol declarar un bloc de quatre columnes per dispositius mitjans o superiors. Els dispositius que no compleixin aquesta regla, és a dir, els dispositius petits o extra reduïts, veuran transformada la classe de forma automàtica a \emph{col-sm-12} o \emph{col-xs-12} respectivament.

    Aquest fet provoca que cada bloc ocupi tot l'espai disponible en una fila (12 columnes) i per tant, que aquests s'apilin un sobre l'altre.
