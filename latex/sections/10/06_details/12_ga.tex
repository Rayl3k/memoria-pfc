\subsection{El fitxer gaTagging.js: Enviant esdeveniments}

    \paragraph{}
    Aquest fitxer s'encarrega de configurar els esdeveniments generals que seran en\-vi\-ats a Google Analytics, però l'aspecte més interessant que conté, és la funció genèrica que s'invoca per enviar esdeveniments des de qualsevol punt de l'aplicació.

    Els esdeveniments de Google Analytics estan formats per quatre paràmetres: \emph{categoria}, \emph{acció}, \emph{etiqueta} i \emph{valor}, on els paràmetres etiqueta i valor són opcionals. El sistema plantejat, ofereix la possibilitat que el paràmetre, \emph{categoria}, sigui omplert automàticament amb el path de la pàgina, que dispara l'esdeveniment, si aquest no és especificat.

    En el bloc de codi que es mostra a continuació, es poden observar dues funcions. La funció, \emph{getCategory}, és l'encarregada d'emplenar el camp \emph{categoria} amb la pàgina i la funció, \emph{sendEvent}, és la funció que s'invoca, de forma centralitzada, per disparar tots els esdeveniments i cridar a la funció \emph{getCategory}, en cas de necessitat.

    \begin{lstlisting}[style=rawOwn,caption={Funcions que controlen l'enviament d'esdeveniments a GA}]
function getCategory() {
    var currentPage = location.pathname.split(`?')[0].slice(1);
    return currentPage = currentPage != `' ? currentPage : `home'
}

function sendEvent(category, action, label, value) {
    category = category != `' ? category : getCategory();
    ...
    ga(`send', `event', category, action, label);
}
    \end{lstlisting}
