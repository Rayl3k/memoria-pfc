\subsection{El fitxer app.js: Funcionament del servidor}

    \paragraph{}
    El fitxer \emph{app.json}, s'encarrega de configurar el servidor i gestionar les peticions HTTP, HTTPS i AJAX provinents del client. És un dels fitxers principals de l'aplicació i probablement un dels més interessants.

    El fitxer es divideix en diferents seccions:

    \begin{itemize}
        \item Creació de variables
        \item Configuració del motor d'impressió i carpetes
        \item Configuració dels complements
        \item Funcions de redirecció
        \item Processament de peticions \emph{POST}
        \item Validació d'identificació
        \item Configuració del servidor
    \end{itemize}


    \subsubsection{Creació de variables}

    \paragraph{}
    En aquesta secció es declarà la utilització dels diferents complements i la creació de l'aplicació mitjançant el framework Express.

    També es creen instàncies dels objectes amb funcions de conveniència em\-mas\-ca\-ra\-des en els fitxers \emph{projectProposals.js}, \emph{pageTitles.js} i \emph{countryParameters.js}.

    \begin{lstlisting}[style=rawOwn,caption={Declaració de variables en el servidor}]
var express = require(`express'),
    app = express(),
    path = require(`path'),
    mustacheExpress = require(`mustache-express'),
    cookieSession = require(`cookie-session'),
    bodyParser = require(`body-parser');

var projectProposals = require(`./assets/js/projectProposals.js');
var projectProposalsIns = new projectProposals();

var pageTitles = require(`./assets/js/pageTitles.js');
var pageTitlesIns = new pageTitles();

var countryParameters = require(`./assets/js/countryParameters.js');
var countryParametersIns = new countryParameters();
    \end{lstlisting}


    \subsubsection{Configuració del motor d'impressió i carpetes}

    \paragraph{}
    Aquest punt esdevé important, ja que indica a l'aplicació quina mena de fitxers ha de renderitzar. En el nostre cas, calia indicar a l'aplicació que els fitxers a pintar en el client eren del format HTML i que s'utilitzava el motor de plantilles Mustache com a complement.

    \begin{lstlisting}[style=rawOwn,caption={Declaració del motor d'impressió}]
app.set(`view_engine', `html');
app.engine(`html', mustacheExpress(`views/globals', `.html'));
    \end{lstlisting}

    \paragraph{}
    A més a més, de cara a què el servidor sigui capaç de resoldre on es troben lo\-ca\-lit\-zats els diferents recursos o fitxers, cal també indicar-ne la ruta de forma relativa al servidor. Així doncs, s'emmascaren les carpetes \emph{views}, \emph{assets}, \emph{node\_modules} i \emph{images} de l'aplicació, de la següent forma.

    \begin{lstlisting}[style=rawOwn,caption={Configuració de les rutes a les carpetes amb fitxers}]
app.set(`views', path.join(__dirname, `views'));
app.use(`/assets', express.static(path.join(__dirname, `assets')));
app.use(`/node_modules', express.static(path.join(__dirname, `node_modules')));
app.use(`/images', express.static(path.join(__dirname, `images')));
    \end{lstlisting}


    \subsubsection{Configuració dels complements}

    \paragraph{}
    En aquest apartat es configuren dos complements. L'encarregat de gestionar les galetes (\emph{cookie-session}) i l'encarregat d'interpretar els paràmetres enviats al servidor des del client (\emph{body-parser}).

    El complement \emph{cookie-session} no cal configurar-lo gaire, ja que els paràmetres per defecte ja marquen les galetes a crear com a segures (transmesa si és possible a través de https) i amb el paràmetre httpOnly activat (no modificable pel client). Per tant, només hem d'indicar-ne el nom i un parell de claus amb les quals vulguem firmar i xifrar el contingut d'aquestes.

    \begin{lstlisting}[style=rawOwn,caption={Configuració del complement \emph{cookie-session}}]
app.use(cookieSession({
    name: `session',
    keys: [`misaholdrin', `tommarvoloriddle']
}));
    \end{lstlisting}

    Per altra banda, el complement \emph{bodyParser}, es preparat per rebre paràmetres de l'URL i descodificar paràmetres JSON.

    \begin{lstlisting}[style=rawOwn,caption={Configuració del complement \emph{bodyParser}}]
app.use(bodyParser.urlencoded({ extended: true }));
app.use(bodyParser.json());
    \end{lstlisting}


    \subsubsection{Funcions de redirecció}

    \paragraph{}
    Aquesta secció del fitxer s'encarrega de decidir i configurar, els fitxers HTML que han de ser retornats segons les peticions realitzades pels usuaris.

    La resposta a totes les peticions GET dels clients són solucionades de forma similar. Primer es capturen, després es recullen tots els paràmetres necessaris per emplenar els paràmetres amb Mustache i finalment, s'envia el HTML processat cap al client.

    \begin{lstlisting}[style=rawOwn,caption={Respota del servidor davant la petició del recurs `/'}]
app.get(`/', function(req, res){
    var params = pageTitlesIns.getTitle(`index');
    var logged = isLogged(req);
    res.render(`index.html', {
        backgroundImage: params[0],
        highlight: params[1],
        title: params[2],
        titleMobile: params[3],
        subtitleDesktop: params[4],
        subtitleTablet: params[5],
        button: params[6],
        buttonHref: `',
        logged
    });
});
    \end{lstlisting}

    El codi anterior representa la resposta a la petició d'accés a la pàgina inicial de l'aplicació (/).

    Un cop es captura la petició GET, denotada per l'URL `/',  es demana a l'objecte Javascript \emph{pageTitlesIns}, carregat a l'inici del servidor, els paràmetres a utilitzar per pintar el fitxer \emph{pageTitle.html} de la pàgina inicial de l'aplicació. Seguidament, es demana si l’usuari es troba identificat amb FamilySearch, mitjançant la funció \emph{isLogged(req)}, que servirà per decidir si cal mostrar o no el botó de desconnexió a la barra de navegació i un cop es reben tots aquests paràmetres, es processa el fitxer, \emph{index.html} amb els paràmetres desitjats i s'envia al client.

    Altres peticions GET més complexes, com per exemple, la de pintar els detalls d'una proposta de projecte específica, funcionen de forma similar, però carregant més paràmetres a part dels de l'objecte \emph{pageTitleIns}. Aquest cas, conté una ex\-cep\-cio\-na\-li\-tat més i és que es captura part de l'URL (\emph{:project}), tal com s'indica en el següent bloc de codi, per utilitzar-la com a paràmetre.

    \begin{lstlisting}[style=rawOwn,caption={Exemple d'utilització com a paràmetre, d'una part de l'URL}]
app.get(`/proposals/:project', function(req, res) {
    var params = pageTitlesIns.getTitle(`index');
    var logged = isLogged(req);
    var proposal = projectProposalsIns.getExample(req.params.project);
});
    \end{lstlisting}


    \subsubsection{Processament de peticions POST}

    \paragraph{}
    Les dues úniques peticions POST que l'aplicació està preparada per acceptar i respondre, són les crides AJAX que serveixen per indicar al servidor que l'usuari s'ha identificat o desconnectat de l'API de FamilySearch.

    Quan es rep una d'aquestes peticions, s'actualitza la galeta de sessió amb el token retornat per l'API de FamilySearch o se n'esborra el contingut, segons la petició tractada. Un cop processada la informació, s'envia un paràmetre \emph{redirect} cap al navegador, que indica a la capa del controlador, a quina pàgina s'ha de redirigir a l'usuari.

    És mostra a continuació, com a exemple, el processament de la petició POST d'identificació.

    \begin{lstlisting}[style=rawOwn,caption={Resposta a la petició AJAX d'identificació}]
app.post(`/token/login', function(req, res) {
    req.session.logged = req.session.logged || req.body.token;
    res.end(`{`redirect' : `/examples'}');
});
    \end{lstlisting}


    \subsubsection{Validació d'identificació}

    \paragraph{}
    La validació d'identificació és una funció que s'utilitza per comprovar si un usuari disposa dels permisos suficients per visitar les pàgines d'exemples implementats.

    Quan el servidor rep la petició de mostrar la pàgina d'exemples o un exemple en concret, realitza una crida a aquesta funció, que resoldrà si la petició ha de ser processada o no. Aquesta funció (\emph{isAuthenticated}), s'insereix com a \emph{middleware} a la capçalera de la petició de la forma seguent:

    \begin{lstlisting}[style=rawOwn,caption={Inserció de \emph{middleware} en una petició del client}]
app.get(`/examples', isAuthenticated, function(req, res) { ... });
    \end{lstlisting}

    La funció, \emph{isAuthenticated}, comprova si la galeta \emph{session} està configurada o no. En cas afirmatiu, es processa la petició inicial de l'usuari i en cas contrari, es mostra la pàgina d'identificació.

    \begin{lstlisting}[style=rawOwn,caption={Comprovació de la galeta \emph{session}}]
function isAuthenticated(req, res, next) {
    if(req.session.isPopulated) next();
    else res.redirect('/login');
}
    \end{lstlisting}

    \paragraph{}
    Al mateix temps, existeix també la funció \emph{isLogged(req)}, que s’utilitza en les crides d’accés a cada pàgina del domini web per comprovar si l’usuari es troba identificat o no amb FamilySearch i retorna un paràmetre que serà utilitzat per decidir si cal mostrar el botó de desconnexió a la barra de navegació.

    \begin{lstlisting}[style=rawOwn,caption={Funció per decidir si cal mostrar el botó de desconnexió}]
function isLogged(req) {
    if(req.session.isPopulated) return 'visible';
    else return 'hidden';
}
    \end{lstlisting}


    \subsubsection{Configuració del servidor}

    \paragraph{}
    La part de configuració del servidor consisteix, principalment, a indicar quin port ha de ser escoltat. La línia de codi utilitzada, permet alternar entre un valor per defecte o l'especificat en els paràmetres de configuració de l'entorn que desplegui el servidor.

    \begin{lstlisting}[style=rawOwn,caption={Configuració del servidor}]
var server = app.listen(process.env.PORT || 8080, process.env.IP, function () { ... });
    \end{lstlisting}
