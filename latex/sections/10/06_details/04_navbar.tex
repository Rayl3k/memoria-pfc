\subsection{El fitxer navbar.html: La barra de navegació adaptativa}

    \paragraph{}
    Com el nom del fitxer indica, aquest s'encarrega de configurar la barra de navegació. Aquesta ha estat creada mitjançant la potencialitat d'un dels components de Bootstrap.

    Consta principalment de dos blocs de codi HTML. El primer, s'utilitza per indicar la capçalera de la barra de navegació, que conté el botó per expandir-la en dispositius mòbils, la icona de l'aplicació que es mostra només en dispositius d'escriptori i el nom de l'aplicació web.

    La capçalera adaptativa de Bootstrap, per la barra de navegació, s'invoca mitjançant la classe \emph{navbar-header}.

    \begin{lstlisting}[style=rawOwn,caption={Capçalera de la barra de navegació}]
<div class=`navbar-header'>
    <button class=`navbar-toggle collapsed'>...</button>
    <a href=`/'><img src=`/images/littleIco.png'/></a>
    <a class=`navbar-brand navbar-link' href=`/'>FamilySearch-PFC</a>
</div>
    \end{lstlisting}

    El segon bloc, s'encarrega de generar els enllaços a les diferents seccions de l'aplicació i configurar el botó dedicat al tancament de la connexió amb FamilySearch, si aquesta està oberta.

    Cada enllaç és estilitzat i configurat mitjançant la classe \emph{navbar-link}. Per decidir si un enllaç ha d'aparèixer per l'esquerra o dreta de la barra de navegació, es mira la inclusió de la classe navbar-righ en el contenidor dels enllaços.

    \begin{lstlisting}[style=rawOwn,caption={Enllaços de la barra de navegació}]
<div class=`collapse navbar-collapse' id=`...'>
	<ul class=`nav navbar-nav'>
		<li><a class=`navbar-link' href=`/background'>Background</a></li>
		<li><a class=`navbar-link' href=`/proposals'>Proposals</a></li>
		<li><a class=`navbar-link' href=`/examples'>Examples</a></li>
	</ul>
	<ul class=`nav navbar-nav navbar-right'>
		<li><a id=`signOut' class=`navbar-link' href>Sign Out</a>
	</ul>
</div>
    \end{lstlisting}
