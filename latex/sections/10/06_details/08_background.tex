\subsection{El fitxer background.html: El grid de Bootstrap II}

    \paragraph{}
    Utilitzarem aquest fitxer per explicar una altra funcionalitat interessant del grid de Bootstrap.

    A causa del fet que aquesta pàgina està formada únicament per text, volem intentar augmentar la llegibilitat d'aquesta. Existeixen diversos estudis que demostren que l'ésser humà llegeix, amb més comoditat, línies curtes que contenen entre 45 i 75 caràcters.

    Per aquest motiu, en aquesta pàgina que tot el contingut principal és text, hem volgut reduir l'amplada màxima utilitzada i en comptes d'utilitzar les dotze columnes que el grid permet, hem decidit utilitzar-ne només vuit.

    Per tal que el contingut no quedi descentrat, és a dir, ocupant vuit columnes des de l'esquerra de la fila i deixant-ne quatre en blanc a la dreta, utilitzem una classe especial de Bootstrap que permet deixar columnes en blanc entre diferents blocs de columnes. Aquesta classe, segueix la forma: \emph{col-SIZE-offset-NUMBER} i es regeix per les mateixes regles que les explicades a l'apartat anterior.

    Això significa que si declarem una regla per dispositius \emph{md}, aquesta no aplicarà per dispositius de grandària inferior. Garantint, que el text segueixi ocupant tota l'amplada possible, en dispositius més petits.

    Així doncs, si volem centrar un contingut de vuit columnes en una fila de dotze columnes, hem de deixar dues columnes en blanc a cada banda del text. La línia de codi que garanteix aquesta visualització és cita a continuació. Fixem-nos, en el fet que primer cal declarar el bloc de columnes i després declarar el \emph{offset}.

    \begin{lstlisting}[style=rawOwn,caption={Separació entre blocs de columnes d'una fila}]
<div class=`row'>
    <div id=`proposal-box-1' class=`col-md-4 col-sm-6'> ... </div>
</div>
    \end{lstlisting}
