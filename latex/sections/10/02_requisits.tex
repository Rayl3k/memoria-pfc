\section{Requisits de l'aplicació web}\label{sec:requisits}

    \paragraph{}
    Aquesta llista pretén oferir un tast dels requisits o manaments que s'han tingut en compte durant el desenvolupament de l'aplicació web.

    \subsection{Requisits funcionals}

    \begin{itemize}
        \item La web ha de permetre identificar-se amb FamilySearch mitjançant el sistema d'autentificació per pop-up.
        \item La web ha de permetre a l'usuari tancar la connexió amb FamilySearch mitjançant una funcionalitat de `Sign Out'.
        \item L'aplicació ha de ser capaç de tancar automàticament la connexió amb Family\-Search si aquesta expira.
        \item La web ha d'incloure una secció que ofereixi un petit resum del rerefons que va originar el projecte.
        \item La web ha de disposar d'una secció en què s'enumerin i exposin les diferents propostes de projecte generades pels futurs estudiants.
        \item L'aplicació ha d'oferir la possibilitat de cercar persones en l'arbre familiar de FamilySearch i observar-ne els detalls d'alguna en concret.
        \item L'aplicació ha de permetre a l'usuari observar l'evolució geogràfica d'un cognom donat un conjunt de països i període de temps.
        \item L'aplicació ha de permetre la visualització del nombre de naixements, casaments i defuncions, enregistrades per un país al voltant d'un any concret.
        \item La secció d'exemples ha de ser només accessible si l'usuari es troba identificat a FamilySearch i ha rebut el token d'ús pertinent.
        \item En cas que el token expiri, l'usuari ha de ser redirigit a la pàgina principal en el moment d'expiració o en la seva següent interacció, si aquest es troba dins de l'àrea d'exemples.
        \item L'aplicació ha d'emmagatzemar el token proporcionat per FamilySearch que rep l'usuari en un recurs que no sigui accessible ni modificable per tercers.
        \item No es permetrà a l'usuari llençar dues crides contra l’API de FamilySearch simultànies per la mateixa funcionalitat des de la mateixa pestanya del navegador.
        \item L’aplicació ha d'aportar la informació bàsica sobre l’origen del projecte i el seu rerefons. L'aplicació també ha d’enllaçar en algun lloc amb el codi font del projecte.
    \end{itemize}


    \subsection{Requisits no funcionals}

    \begin{itemize}
        \item L'aplicació web ha de funcionar i ser visualitzada de forma correcta en els principals navegadors web moderns.
        \item Els formularis de l'aplicació web que puguin generar errors han de proporcionar informació bàsica a l'usuari en el moment que el camp és abandonat o informació més detallada si envia el formulari amb errors.
        \item Els formularis de l’aplicació han de donar un feedback positiu en cas que els camps siguin omplerts de forma correcta.
        \item La web ha de ser relativament fàcil d'utilitzar, oferint les eines necessàries als usuaris i facilitant la navegació per les diferents seccions.
        \item Mentre l'aplicació web espera resposta de l’API de FamilySearch, s'ha de mostrar a l'usuari que l'aplicació es troba esperant resultats i el progrés rea\-lit\-zat fins al moment.
        \item L'aplicació ha de donar un feedback clar a l'usuari quan la interacció amb l’API de FamilySearch finalitza.
        \item L'aplicació ha de ser navegable de forma acceptable mitjançant dispositius mò\-bils. Les integracions amb l’API de FamilySearch també han de ser utilitzables, però no cal que la informació retornada es trobi completament adaptada.
        \item Les imatges de l'aplicació s'han de trobar optimitzades en la mesura que sigui possible per intentar que aquesta carregui el més ràpid possible en un entorn d'hostalatge gratuït.
        \item Les imatges principals de l'aplicació s'han de carregar de formar transparent quan l'aplicació és iniciada i la primera pàgina és carregada, per millorar la visualització de la web quan l'usuari navega entre les diferents seccions.
        \item La llengua utilitzada en el web serà l'anglès per tal d'ajudar i facilitar el procés de certificació.
        \item El projecte s'ha de trobar sota una certificació Creative Commons d'atribució no comercial.
        \item La informació sobre el codi font del projecte, la llicència i la facultat d'in\-for\-mà\-ti\-ca, ha de trobar-se disponible en el peu de pàgina de les pàgines de l'aplicació.
        \item Es podrà monitorar la navegació dels usuaris pel web, així com les seves accions principals i errors generats.
        \item Es podrà obtenir la configuració dels sistemes amb els quals s'ha navegat per la web i veure si el comportament d'algun d'ells és més propici a la generació d'errors.
        \item Els fitxers Javascript s'han de trobar el més al final possible dels arxius HTML per facilitar la càrrega del contingut.
        \item Es reutilitzarà codi HTML i Javascript en la mesura que sigui possible per tal d'evitar la duplicació de contingut.
    \end{itemize}
