\section{Accés a l'arbre familiar i operacions de cerca}

    \paragraph{}
    L'accés a les dades contingudes per l'API de FamilySearch, pot ser realitzat de diferents formes segons la informació coneguda en el moment d’iniciar la cerca.

    Tot procés de cerca pot ser dividit en dues fases principals, on la segona, realment consisteix en un conjunt d'opcions diferents, que marcaran com les dades seran accedides. Les dues fases principals consisteixen en:

    \begin{enumerate}
        \item Accedir al sistema de FamilySearch mitjançant un usuari i contrasenya.
        \item Accedir a les dades de forma directa o indirecta, segons la informació coneguda.
    \end{enumerate}

    La imatge~\ref{fig:dataAcessPath}, ofereix una vista preliminar de les diferents opcions disponibles, per tal d'accedir a les dades de FamilySearch, relacionades amb les persones de l'arbre familiar. En els següents apartats, s'exposarà el funcionament dels mètodes d'accés a les dades directes i indirectes.

    \begin{figure}[h]
        \includegraphics[width=\linewidth]{05/08_dataAccessPaths}
        \centering
        \caption{Sistemes d'accés a les dades de l'arbre familiar}\label{fig:dataAcessPath}
    \end{figure}


    \subsection{L'accés directe a les dades}

        \paragraph{}
        L'accés directe a les dades es pot realitzar si es coneix l'identificador del recurs que vol ser consultat.

        Per exemple, si es coneix el PID de la persona a consultar, es pot codificar des de les aplicacions el diferent conjunt d'URIs necessàries per manipular els recursos i evitar així, la necessitat de passar pel sistema.

        D'aquesta forma, es pot accedir directament a la informació específica dels recur\-sos Persona, Memòries, Discussions, Fills, Pares, Parelles, Avantpassats i Descen\-dència; sempre i quan es conegui tota la informació necessària per codificar les URIs d'accés, de la persona a consultar.

        Donat un identificador de persona conegut, un exemple d'ús, podria ser per exemple, accedir directament a les memòries de la persona mitjançant la codificació de la URI i evitar així, haver de passar primer per l'arbre familiar.


    \subsection{L'accés indirecte a les dades}

        \paragraph{}
        A pesar que accedir directament a les dades pot semblar un procés ideal, el més normal és que ens trobem en una situació en què desconeguem totes les dades necessàries per codificar les URIs i que per tant, sigui necessari realitzar un accés indirecte a les dades.

        En altres paraules, accedir en primera instància als recursos del sistema i des d'aquests, obtenir els diferents enllaços que ens condueixin cap a les operacions i informació desitjada.

        Per descobrir l’URI del recurs exacte que vol ser consultat, existeixen principalment tres opcions diferents:

        \begin{itemize}
            \item Llegir la col·lecció Arrel, que representa la col·lecció que conté totes les dades de FamilySearch i seguir-la per tal d'obtenir l'usuari identificat. Un cop es disposa de la persona relacionada a l'usuari, es pot accedir a qualsevol altre recurs relacionat mitjançant els enllaços hypermedia de la resposta i simular, de forma dinàmica, el mateix que podríem haver realitzat mitjançant l'accés directe.
            \item La segona opció i probablement, la més comuna, passa per la realització d'una cerca contra la base de dades de FamilySearch per tal d'obtenir una Persona o Localització. Un cop s'obté el resultat de la cerca, es pot navegar a qualsevol de les peces d'informació relacionades als recursos retornats, mitjançant les URI i enllaços hypermedia de les respostes.
            \item Finalment, l'última opció consisteix a llegir la llista de col·leccions disponibles per tal d'accedir a l'arbre familiar. Un cop es disposa de l'arbre familiar, es pot llegir l'usuari actual o la persona relacionada a l'usuari i des d'aquí procedir com es desitgi. Una altra alternativa, és accedir mitjançant els enllaços o plantilles URL, als recursos generals de les Memòries, Discussions, Relacions, etcètera i consultar la informació general d'aquests recursos mitjançant els seus identificadors retornats.
        \end{itemize}

        El procés descrit en el segon punt de les opcions de cerca indirecta, serà el més comú de cara a accedir a la informació de l'arbre familiar, ja que les altres opcions no permeten gaire filtratge de les dades, a menys que coneguem la col·lecció exacte sobre la qual volem realitzar una investigació genealògica o només vulguem consultar la informació de l'arbre familiar, relativa a l'usuari identificat.

        Per tant, com hem comentat, són tres les principals opcions d'entrada indirecta a l'arbre familiar, que permeten explorar el conjunt de dades genealògiques de les diferents persones i localitzacions.

        En els següents apartats, s'explicarà en més detall com funcionen les principals operacions de lectura indirectes.


    \subsection{Lectura de l'usuari identificat}

        \paragraph{}
        Aquesta operació pot ser realitzada de forma directe, accedint a la URI:

        \begin{displayquote}
            \emph{/platform/users/current}
        \end{displayquote}

        La crida retorna la informació específica del recurs Usuari descrita en l'apar\-tat~\ref{sec:user} d'aquesta memòria. Des d'aquest, es pot accedir al recurs de la persona que re\-pre\-sen\-ta a l'usuari i des d’aquest, a qualsevol altre peça d'informació genealògica relacionada amb la persona.


    \subsection{Lectura de la persona relacionada a l'usuari identificat}

        Aquesta operació és realitzable a través de la URI:

        \begin{displayquote}
            \emph{/platform/tree/current-person}
        \end{displayquote}

        L'operació en qüestió retorna el recurs Persona de l'usuari identificat, amb tota la informació descrita en l'apartat~\ref{sec:persons} de la memòria. Des d'aquest, es pot navegar a qualsevol altre recurs o dada genealògica relacionada.

        Aquesta forma d'accés representa un pas menys que l’exposada en l’apartat anterior, si tenim clar des del principi que l'objectiu final és accedir a l'arbre familiar. En cas que també volguéssim alguna dada del recurs Usuari, caldria entrar al sistema per la funcionalitat descrita en l'apartat anterior.


    \subsection{Cerca de Persones a l'arbre familiar}\label{sec:searchParams}

        \paragraph{}
        L'operació cerca de persones a l'arbre familiar és sense cap mena de dubte, la més interessant de totes les funcionalitats que ofereix l'API.

        Aquesta operació es realitza mitjançant una petició a la URI \emph{/platform/tree/\-search}, que a més a més, pot ser configurada i personalitzada mitjançant la inclusió de di\-fe\-rents paràmetres.

        Un cop finalitzada la petició, aquesta retorna el conjunt de persones de l'arbre familiar que compleixen amb les condicions imposades en la cerca. Des d'aquestes, l'usuari pot començar a navegar per la resposta, accedint a aquelles persones que li resultin de més interès i a tots els altres recursos disponibles, relacionats amb aquestes.

        Es pot entendre la funcionalitat de cerca de persones a l'arbre familiar, com una porta a totes les dades emmagatzemades per FamilySearch.

        La cerca pot ser controlada mitjançant tres paràmetres principals, un dels quals, accepta molts paràmetres secundaris. Els paràmetres principals són descrits a la taula~\ref{res:searchPersonMain}.

        \begin{center}
                 \csvreader[
                    separator=semicolon,
                    before table=\sffamily\small,
                    respect tilde=true,
                    respect leftbrace=true,
                    respect rightbrace=true,
                    longtable={p{2cm-2\tabcolsep}p{12cm-2\tabcolsep}},
                    table head={\caption{Variables principals per la cerca de persones}\label{res:searchPersonMain}\\\toprule%
                        \headentry{m{2cm-2\tabcolsep}}{Variable}
                        & \headentry{m{12cm-2\tabcolsep}}{Descripció}\\\midrule},
                    late after line=\\\midrule,
                    late after last line=\\\bottomrule,
                 ]
                 {./tables/05/10_search/searchPersonMain.csv}
                 {var=\var,desc=\desc}
                 {\var&\desc}
         \end{center}

         Per altra banda, el paràmetre \emph{q} descrit en la taula~\ref{res:searchPersonMain}, accepta com a paràmetres vàlids els camps exposats a la taula~\ref{res:searchPersonSec}. L'etiqueta \emph{\{relation\}} dels camps d'aquesta taula, pot ser reemplaçada per qualsevol dels següents valors: \emph{father}, \emph{mother}, \emph{spouse} \emph{(pare, mare, parella)}.

         \begin{center}
                  \csvreader[
                     separator=comma,
                     before table=\sffamily\small,
                     respect tilde=true,
                     respect leftbrace=true,
                     respect rightbrace=true,
                     longtable={p{4cm-2\tabcolsep}p{10cm-2\tabcolsep}},
                     table head={\caption{Paràmetres acceptats per la variable q}\label{res:searchPersonSec}\\\toprule%
                         \headentry{m{4cm-2\tabcolsep}}{Variable}
                         & \headentry{m{10cm-2\tabcolsep}}{Descripció}\\\midrule},
                     late after line=\\\midrule,
                     late after last line=\\\bottomrule,
                  ]
                  {./tables/05/10_search/searchPersonSec.csv}
                  {param=\param,desc=\desc}
                  {\param&\desc}
          \end{center}


          \subsubsection{Cerca de persones duplicades}

          \paragraph{}
          Una característica interessant de les respostes en la cerca de persones és que per cada persona de la resposta, es pot accedir al conjunt de persones de l'arbre familiar, que amb una alta probabilitat, poden representar a la mateixa persona consultada. En altres paraules, persones que podrien tractar-se de possibles duplicats.

          Aquesta funcionalitat de cerca de persones duplicades, també es troba accessible de forma directa, si es coneix l'identificador personal de la persona sobre la qual es volen cercar els possibles duplicats. Per exemple:

          \begin{displayquote}
              \emph{/platform/tree/persons/{pid}/matches}
          \end{displayquote}


    \subsection{Cerca de localitzacions}

        \paragraph{}
        La cerca de localitzacions és la segona opció de cerca massiva que permet l'API de FamilySearch.

        Aquesta operació cobra especial interès quan es necessita consultar informació extra sobre una localització, més enllà de la informació bàsica relacionada a certs recursos de l'arbre familiar o es vol obtenir informació sobre totes les localitzacions  que compleixen amb certs criteris. Per exemple, totes les localitzacions dins d'una jurisdicció específica.

        Així doncs, la cerca de localitzacions permet relacionar o interpretar, el nom d'una localització, mitjançant una descripció estandarditzada a la URI:

        \begin{displayquote}
            \emph{/platform/places/search}
        \end{displayquote}

        De la mateixa forma que en la cerca de persones, la cerca de localitzacions pot ser configurada mitjançant tres paràmetres principals, on un d'aquests accepta diversos paràmetres secundaris. Els paràmetres principals són descrits a la taula~\ref{res:searchLocMain}.

        \begin{center}
                 \csvreader[
                    separator=semicolon,
                    before table=\sffamily\small,
                    respect tilde=true,
                    respect leftbrace=true,
                    respect rightbrace=true,
                    longtable={p{2cm-2\tabcolsep}p{12cm-2\tabcolsep}},
                    table head={\caption{Variables principals per la cerca de localitzacions}\label{res:searchLocMain}\\\toprule%
                        \headentry{m{2cm-2\tabcolsep}}{Variable}
                        & \headentry{m{12cm-2\tabcolsep}}{Descripció}\\\midrule},
                    late after line=\\\midrule,
                    late after last line=\\\bottomrule,
                 ]
                 {./tables/05/10_search/searchLocMain.csv}
                 {var=\var,desc=\desc}
                 {\var&\desc}
         \end{center}

         En aquesta operació, el paràmetre \emph{q} accepta com a vàlids el conjunt de paràmetres especificats a la taula~\ref{res:searchLocSec}.

         \begin{center}
                  \csvreader[
                     separator=semicolon,
                     before table=\sffamily\small,
                     respect tilde=true,
                     respect leftbrace=true,
                     respect rightbrace=true,
                     longtable={p{4cm-2\tabcolsep}p{10cm-2\tabcolsep}},
                     table head={\caption{Paràmetres acceptats per la variable q}\label{res:searchLocSec}\\\toprule%
                         \headentry{m{4cm-2\tabcolsep}}{Variable}
                         & \headentry{m{10cm-2\tabcolsep}}{Descripció}\\\midrule},
                     late after line=\\\midrule,
                     late after last line=\\\bottomrule,
                  ]
                  {./tables/05/10_search/searchLocSec.csv}
                  {param=\param,desc=\desc}
                  {\param&\desc}
          \end{center}
