\subsection{El recurs Subjecte (Subject)}

    \paragraph{}
    El recurs Subjecte fa referència al concepte abstracte de subjecte genealògic, entenent-lo com una entitat única que o bé pot fer referència a una persona o a una localització concreta sobre el globus terraqüi.

    Aquesta entitat única, agrega i emmagatzema les diferents evidències i fonts de dades particulars del subjecte, és a dir, el col·lectiu d'informació que fan que aquest subjecte sigui diferent dels altres.

    La taula~\ref{res:subject} mostra al que ens referim quan parlem d'aquesta agregació d'informació.

    \begin{center}
             \csvreader[
                separator=comma,
                before table=\sffamily\small,
                longtable={p{2cm-2\tabcolsep}p{3.5cm-2\tabcolsep}p{8.5cm-2\tabcolsep}},
                table head={\caption{Paràmetres del recurs Subjecte}\label{res:subject}\\\toprule%
                    \headentry{m{2cm-2\tabcolsep}}{Paràmetre}
                    & \headentry{m{3.4cm-2\tabcolsep}}{Format de Dades}
                    & \headentry{m{8.5cm-2\tabcolsep}}{Descripció}\\\midrule},
                late after line=\\\midrule,
                late after last line=\\\bottomrule,
             ]
             {./tables/05/06_others/subject.csv}
             {param=\param,format=\format,desc=\desc}
             {\param&\format&\desc}
     \end{center}
