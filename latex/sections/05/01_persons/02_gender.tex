\subsection{El recurs Gènere (Gender)}

    \paragraph{}
    El recurs Gènere s'utilitza per especificar el gènere d'una persona en concret. Aquest recurs conté els paràmetres propis mostrats a la taula~\ref{tab:gender} i hereta els camps de dades dels recursos Conclusió, Enllaços Hypermedia i Dades Extensibles que poden ser trobats a la secció `Altres recursos interessants'.

    \begin{center}
             \csvreader[
                separator=comma,
                before table=\sffamily\small,
                longtable={p{2cm-2\tabcolsep}p{3.5cm-2\tabcolsep}p{8.5cm-2\tabcolsep}},
                table head={\caption{Codificació GEDCOM X del recurs Gènere}\label{tab:gender}\\\toprule%
                    \headentry{m{2cm-2\tabcolsep}}{Paràmetre}
                    & \headentry{m{3.4cm-2\tabcolsep}}{Format de Dades}
                    & \headentry{m{8.5cm-2\tabcolsep}}{Descripció}\\\midrule},
                late after line=\\\midrule,
                late after last line=\\\bottomrule,
             ]
             {./tables/05/01_persons/gender.csv}
             {param=\param,format=\format,desc=\desc}
             {\param&\format&\desc}
     \end{center}


     \subsubsection{L'enumeració genderType}

     \paragraph{}
     L'enumeració genderType segueix l'estructura de definició GEDCOMX. Com a tal, els valors possibles per l'enumeració segueixen la pauta:

     http://gedcomx.org/ + `genderType'

     La següent taula mostra els tres possibles valors de l'enumeració genderType.

     \begin{center}
         \csvreader[
            no head,
            separator=comma,
            table head={\caption{Valors possibles per l'enumeració genderType}\label{tab:genderType}},
            before table=\sffamily\small,
            longtable={|p{3cm}|p{3cm}|p{3cm}|},
            column count=4,
            late after head=\\\hline,
            late after line=\\\hline,
            late after last line=\\\hline,
         ]
         {./tables/05/01_persons/genderType.csv}
         {1=\one,2=\two,3=\three}
         {\one&\two&\three}
     \end{center}
