\section{Recursos principals del bloc Discussions}

    \paragraph{}
    El recurs principal d'aquest bloc de l'arbre familiar rep el nom de Discussió. Al mateix temps, podem dir que les discussions estan formades principalment per un conjunt de Comentaris.

    Les discussions a FamilySearch són tòpics de discussió introduïts pels usuaris i relacionats a una persona en concret. Com hem comentat, aquestes discussions estan formades per diferents comentaris i també destaca la utilització d'un recurs intermedi que fa de pont entre les discussions i els recursos de les persones a les quals fan re\-fe\-rèn\-cia, mitjançant enllaços hypermedia.

    El contingut d'aquestes discussions és bastant divers, però generalment són u\-ti\-lit\-za\-des per discutir, entre diferents usuaris, sobre les dades relatives a una persona, l'origen de les fonts de dades i altres aspectes similars.

    La figura~\ref{img:discussionsBloc} mostra com es troben relacionats els recursos que conformen el bloc Discussions.

    \begin{figure}[h]
        \includegraphics{05/05_discussionsCore}
        \centering
        \caption{El bloc de l'arbre familiar relatiu a les discussions}\label{img:discussionsBloc}
    \end{figure}

    Podreu observar també, a les taules que representen l'estructura dels recursos, que a vegades, per la columna que marca el format de dades d'un paràmetre, aquest es troba especificat entre els caràcters `[' i `]'. Aquesta terminologia s'utilitza per indicar que aquest paràmetre és en realitat un recurs o objecte de dades diferent inclòs dins del recurs estudiat.

    També s'observarà que sovint, els recursos exposats, hereten dades d'altres recursos i en els casos que aquests siguin rellevants, se n'exposarà l'estructura a l'apartat `Altres recursos interessants', més endavant en la memòria.

    \input{./sections/05/03_discussions/01_discussionReference}
    \input{./sections/05/03_discussions/02_discussion}
    \input{./sections/05/03_discussions/03_comment}
