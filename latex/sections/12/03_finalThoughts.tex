\section{Últimes consideracions}

    \paragraph{}
    Crec que resulta raonable dir, a pesar de totes les coses que m’agradaria canviar de disposar de l’oportunitat de tornar a començar de zero, que el projecte ha resultat satisfactori respecte els objectius que aquest s’havia marcat.

    És cert que em quedo amb una certa sensació de què el resultat podria ser millor de poder realitzar alguna iteració més sobre el projecte, però al final, un projecte disposa d’un temps de desenvolupament limitat i no totes les ambicions poden quedar satisfetes.

    Donat l’esforç i temps invertit, tant en l’estudi de l’API com en les tecnologies per implementar l’aplicació web, estic molt satisfet amb els resultats obtinguts en els apartats teòrics i pràctics del projecte.

    Al mateix temps, si recordem que aquest projecte partia d’uns objectius definits, però incerts de cara a conèixer la profunditat o qualitat dels possibles resultats, les propostes de projecte ofertes com a base pels futurs estudiants i els exemples implementats, resulten prou atractius i prometedors, per justificar l’esforç invertit i estar contents pel resultat obtingut.

    A part, el fet d'haver aconseguit certificar l'aplicació i que FamilySearch ens publiqui el codi al costat de les aplicacions d'exemple oficials, és un reconeixement a la feina feta i a què el projecte ha aconseguit complir amb els objectius didàctics que ens havíem marcat per ell.

    També és cert, que m’hagués agradat poder plantejar una altra mena de propostes de projecte, si l’API així ho hagués permès, però de totes maneres, no es pot descartar l’opció que l’API evolucioni en un futur pròxim i permeti operacions més complexes  i profundes sobre ella.

    Pel que fa a la planificació presentada a la secció~\ref{sec:thePlan}, aquesta ja havia estat adaptada a la realitat que havia conformat el projecte, 

    Finalment, comentar, que en l’àmbit personal, aquest projecte m’ha permès obtenir aquells coneixements que m’havia marcat com a repte personal i relacionar-lo, en major o menor mesura, a la meva vida laboral com a analista de dades.

    Al mateix temps, el projecte m’ha fet créixer en algun aspecte personal, que no m’esperava. En concret, l’exigència de compaginar una jornada laboral completa amb l’elaboració d’un projecte com aquest, m’ha portat a l’esgotament en més d’una ocasió. Certs problemes trobats en el camí, semblaven obstacles insuperables i ha estat gràcies al suport rebut per diferents persones del meu entorn, que he aconseguit no abandonar, seguir treballant i finalitzar aquest projecte; des del meu punt de vista, de forma satisfactòria.
