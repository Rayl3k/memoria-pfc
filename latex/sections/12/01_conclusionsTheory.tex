\section{Conclusions sobre el bloc teòric}

    \paragraph{}

    En condicions normals, aquest bloc del projecte hauria estat estudiat i realitzat en profunditat al començament del projecte. No obstant això, a causa de la voluntat inicial d’explorar el procés de certificació, ens vam veure obligats a concentrar-nos en el desenvolupament de l’aplicació tan aviat com fos possible i es va decidir dividir, l’estudi teòric, en dues parts.

    La primera, un breu però fort estudi enfocat a estudiar les diferents formes de connectar-se a l’API i els principals paràmetres accessibles. D’aquesta forma, s’obtenia un coneixement inicial suficient per guiar el rumb de la implementació d’exemples.

    La segona part de l’estudi va succeir un cop el desenvolupament de l’aplicació es trobava força avançat.  Arribats a aquest punt, es va començar a estudiar en detall el conjunt d’informació accessible per l’API i com aquesta estava relacionada, per quins països es disposava de més informació o registres, les limitacions d’utilització i altres aspectes similars.

    Podríem dir, que aquest segon estudi és el que ens va permetre acabar de comprendre, en total profunditat, com estaven connectades les diferents peces d’informació, els diferents camins d’accés possibles cap a les dades i les necessitats per les quals l’API havia estat dissenyada.

    A pesar de disposar de la documentació oficial de FamilySearch sobre l’API, adquirir aquest coneixement no va resultar una tasca fàcil. La informació no es trobava realment exposada, actualitzada ni ordenada i va fer falta posar ordre en el caos, per tal de poder oferir un dibuix comprensible sobre el funcionament d’aquesta.

    El fet de dividir l’explicació de l’API, en els diferents grans blocs de recursos i camins d’accés, tot mostrant les principals dades genealògiques contingudes pels recursos de cada bloc, facilita en gran mesura la comprensió de l’estructura de l’API, la jerarquia de la informació i els camins d’accés.

    És per aquest motiu, que estic bastant content amb el resultat obtingut i crec que la lectura de la secció destinada a l’estudi en profunditat de l’API, suposa un benefici pels nou vinguts que vulguin comprendre com aquesta funciona i quina informació emmagatzema, tot reduint, l’esforç necessari d’obtenció, d’un coneixement que no resulta tan accessible.

    Per tant, puc opinar ara, en el moment de finalitzar la memòria, que s’ha aconseguit assolir de forma satisfactòria el primer gran objectiu del projecte, que recordem, consistia a estudiar i exposar la informació accessible a través de l’API i com aquesta es trobava relacionada.

    Pel que fa al plantejament de propostes de projecte, el segon estudi que es va realitzar sobre l’API va esdevenir crucial a l’hora de plantejar un conjunt de propostes variades, de diferents dificultats i que oferís projectes relativament interessants.

    És realment una pena, que aquest estudi més profund, acabes exposant que la major part dels registres accessibles a través de l’API (62\%), pertanyés als Estats Units. Aquest fet ens deixava bastant lligats de mans a l’hora de plantejar projectes globals o particulars a una regió, que per proximitat geogràfica, ens hagués pogut resultar més interessant.

    També va tenir un impacte en la qualitat d’aquest apartat de la memòria, el fet que l’API hagués estat construïda primerament i de forma exclusiva, per la cerca d’avantpassats. Això limita en gran mesura les operacions que poden ser realitzades contra l’API o els volums d’informació descarregable pel seu anàlisi. Bàsicament, tot intent de mineria de dades, queda impossibilitat.

    Tot i l’existència d’un acord de col·laboració, que permet a organitzacions genealògiques descarregar, al seu propi sistema, còpies de les dades de FamilySearch, crec que aquest no està pensat per estudiants i per tant, no semblava honest jugar aquesta carta per tal de poder posar sobre la taula propostes de projecte més interessants.

    De totes maneres, donades les regles de joc marcades per FamilySearch, opino que s’ha aconseguit crear un conjunt de propostes prou interessant. En concret, s’han intentat oferir tres vies de projecte diferents:

    \begin{itemize}
        \item \textbf{Explotació de les dades existents:} Propostes com els estudis específics de la segona guerra mundial o l’elaboració més en profunditat de la funcionalitat geolocalització de cognoms, pretenen explotar el conjunt de dades més ric contingut a l’API de FamilySearch i crear així funcionalitats que explotin la informació accessible i intentin desemmascarar patrons en les dades.
        \item \textbf{Propostes independents del contingut:} Propostes com la implementació d’un portal en català, relacionar els cognoms de l’API amb l’heràldica o propostes sobre el procés d’indexació, tenen com a objectiu plantejar un seguit de propostes més enfocades al desenvolupament d’una funcionalitat usable, que en desemmascarar patrons amagats a les dades de FamilySearch.
        \item \textbf{Futures investigacions de potencialitat:} Aquest projecte no ha estat capaç d’explorar, en total profunditat, la potencialitat de l’API en tots els àmbits. Per aquest motiu, s’han plantejat un seguit de propostes com poden ser la comparació de dades genealògiques reals amb FamilySearch o l’estudi de les col·leccions accessibles, perquè altres estudiants puguin contribuir, en l’estudi de potencialitat, per aquelles regions menys riques en nombre de registres.
    \end{itemize}

    Hem volgut oferir també aquestes tres vies diferents d’exploració, com a catalitzador perquè els futurs estudiants, no es limitin només a la realització d’alguna de les propostes, sinó que aquestes siguin utilitzades com a inspiració per abordar altres preguntes, que els hi puguin resultar més atractives en l’àmbit personal.

    Per tancar les conclusions dels dos primers grans objectius del projecte, podríem dir que crec que aquests han estat assolits de forma satisfactòria, a pesar de ser una llàstima, no haver pogut plantejar propostes de projecte més interessants donada la naturalesa de l’API i les restriccions imposades per aquesta. Podríem doncs concloure, que l’API ofereix menys potencial del que hauríem desitjat en el moment d’iniciar el projecte, però que tot i això, diferents projectes interessants poden néixer a partir d’aquesta.
