\section{Conclusions sobre el bloc pràctic}

    \paragraph{}
    Com s’ha esmentat en les conclusions de l’apartat teòric, aquest projecte va abordar la implementació de l’aplicació web tan aviat com va resultar possible, per poder optar a explorar el procés de certificació.

    Aquest fet ha suposat, que mirant el projecte en retrospectiva i amb tot el coneixement adquirit a l’esquena, lamentem alguna de les decisions preses pel que fa a l’arquitectura de l’aplicació web o als exemples que han estat implementats.

    En primer lloc, després de realitzar tota la implementació de l’aplicació web, un es dóna compte que haver implementat tota la lògica de la interacció amb l’API de FamilySearch, en el servidor, en comptes del controlador, no hagués suposat un esforç tan elevat com es creia i en contrapartida, hauríem obtingut una aplicació més robusta, segura i escalable.

    En segon lloc, haver pogut realitzar l’estudi exhaustiu de l’API des del principi, hauria permès, per exemple, que la funcionalitat d’expansió geogràfica d’un cognom, estudies els moviments d’un cognom entre els diferents estats dels Estats Units, aprofitant l’alt volum de registres disponibles per aquest país, en comptes de l’evolució geogràfica a través de diferents països.

    De totes maneres, a pesar de totes les coses que faríem diferent si disposéssim de l’oportunitat de començar de nou el projecte, crec que l’aplicació web desenvolupada compleix amb els objectius que ens havíem marcat per aquesta.

    Les tres funcionalitats implementades, ofereixen un bon tast del tipus de projectes que poden ser realitzats i exposen la informació genealògica més important, accessible sobre les diferents persones emmagatzemades a l’arbre familiar. Per aquest motiu, crec que les funcionalitats implementades compleixen amb l’objectiu establert i sempre poden ser millorades, ampliades i concretades, pels futurs estudiants.

    Personalment, estic molt content amb el resultat de la implementació del web, ja que han estat moltes i diverses les tecnologies a estudiar i fer treballar de forma conjunta. Visualment, crec que també s’ha aconseguit un bon resultat i és que la quantitat d’hores invertides, en fer que la web es vegi i comporti de la forma en què ho fa, per tota mena de dispositius diferents, resulta molt difícil de quantificar.

    A l’apartat 10.2, es descrivia el conjunt de requisits funcionals i no funcionals que la web havia de complir i opinió que aquests han estat assolits en gran mesura.

    A l’apartat 10.2, es descrivia el conjunt de requisits funcionals i no funcionals que la web havia de complir i opinió que aquests han estat assolits en gran mesura.

    Tot i que evidentment, molts aspectes d’usabilitat són millorables, com podria ser per exemple la facilitació de la navegació vertical per les funcionalitats implementades o la visualització dels resultats de la funcionalitat de cerca en dispositius mòbils, crec que en general s’han aplicat molts conceptes de disseny i usabilitat que faciliten l’ús i comprensió de l’aplicació. Acabar de refinar-la, suposa només una qüestió d’inversió d’hores, doncs el coneixement tècnic ja ha estat adquirit.

    Si repassem una mica cada una de les funcionalitats, opino que la funcionalitat de cerca de persones, representa un bon punt de partida per aquelles persones interessades a descobrir la informació genealògica disponible o en com la cerca de persones pot ser configurada.

    La funcionalitat evolució geogràfica d’un cognom, representa una molt bona idea, que per desgràcia, donada la naturalesa de les dades accessibles, no ha acabat de resultar com m’esperava. De totes maneres, possibles iteracions sobre aquesta són possibles i així ho hem reflectit en les propostes de projecte per futurs estudiants. Per tant, tot i no resultar una funcionalitat tan satisfactòria com pensàvem que podria ser, sí que compleix amb el rol principal que tenia marcada, d’exposar el potencial de l’API.

    Finalment, la funcionalitat d’evolució d’esdeveniments, tot i resultar la més simple de les tres, també és, des del meu punt de vista, la funcionalitat que permet explotar de forma més eficaç les dades emmagatzemades per l’API i produir resultats més satisfactoris. Considero que aquesta funcionalitat representa un bon punt de partida per comparar els registres de diferents països o plantejar la viabilitat de certs projectes.

    Pel que fa al procés de certificació, a pesar del caos que aquest va resultar al principi amb tots els entrebancs que es van patir intentant iniciar el procés de certificació, des del moment que es va aconseguir contactar a l’organització per correu ... [fill depending on ending]
