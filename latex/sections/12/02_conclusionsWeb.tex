\section{Conclusions sobre el bloc pràctic}

    \paragraph{}
    Com s’ha esmentat en les conclusions de l’apartat teòric, aquest projecte va abordar la implementació de l’aplicació web, tan aviat com va resultar possible, per tenir temps d'explorar el procés de certificació.

    Aquest fet ha suposat, que mirant el projecte en retrospectiva i amb tot el conei\-xe\-ment adquirit sobre les espatlles, lamentem alguna de les decisions preses pel que fa a l’arquitectura de l’aplicació web o als exemples que han estat implementats.

    En primer lloc, després de realitzar tota la implementació de l’aplicació web, un es dóna compte que haver implementat tota la lògica de la interacció amb l’API de FamilySearch, en el servidor, en comptes del controlador, no hagués suposat un esforç tan elevat com es creia i en contrapartida, hauríem obtingut una aplicació més robusta, segura i escalable.

    En segon lloc, haver pogut realitzar l’estudi exhaustiu de l’API des del principi, hauria permès, per exemple, que la funcionalitat d’expansió geogràfica d’un cognom, estudies els moviments d’un cognom entre els diferents estats dels Estats Units, aprofitant l’alt volum de registres disponibles per aquest país, en comptes de l’evolució geogràfica a través de diferents països.

    De totes maneres, a pesar de totes les coses que faríem diferent si disposéssim de l’oportunitat de començar de nou el projecte, crec que l’aplicació web desenvolupada, compleix amb els objectius que ens havíem marcat per aquesta.

    Les tres funcionalitats implementades, ofereixen un bon tast del tipus de projectes que poden ser realitzats i exposen la informació genealògica més important, accessible sobre les diferents persones que conformen l’arbre familiar. Per aquest motiu, crec que les funcionalitats implementades compleixen amb l’objectiu establert i sempre poden ser millorades, ampliades i concretades, pels futurs estudiants.

    Personalment, estic molt content amb el resultat de la implementació del web, ja que han estat moltes i diverses les tecnologies a estudiar i fer treballar de forma conjunta. Visualment, crec que també s’ha aconseguit un bon resultat i és que la quantitat d’hores invertides, en fer que la web es vegi i comporti de la forma en què ho fa, per tota mena de dispositius diferents, resulta molt difícil de quantificar.

    A l'apartat~\ref{sec:requisits}, es descrivia el conjunt de requisits, funcionals i no funcionals, que la web havia de complir i opinió, que aquests han estat assolits de forma satisfactòria.

    Molts dels requisits no funcionals, giraven al voltant de la usabilitat de l'aplicació web. Si recordem una de les motivacions que ens va portar a fer aquest projecte, era que aquest ens permetria posar en pràctica, el coneixement obtingut en l'etapa d'interí, com a dissenyador d'experiència d'usuari.

    Tot i que evidentment, molts aspectes d'usabilitat són millorables, com podria ser per exemple la facilitació de la navegació vertical per les funcionalitats implementades o la visualització dels resultats de la funcionalitat de cerca en dispositius mòbils, crec que en general s'han aplicat molts conceptes de disseny i usabilitat, que faciliten l'ús i comprensió de l'aplicació. Acabar de refinar-la, seria una qüestió d'inversió d'hores, doncs el coneixement tècnic necessari, ja ha estat adquirit.

    Si repassem una mica cada una de les funcionalitats implementades, opino que la funcionalitat de cerca de persones, representa un bon punt de partida, per aquelles persones interessades a descobrir la informació genealògica disponible o en com la cerca de persones pot ser configurada.

    La funcionalitat evolució geogràfica d'un cognom, representa una molt bona idea, que per desgràcia, donada la naturalesa de les dades accessibles, no ha acabat de resultar com m'esperava. De totes maneres, possibles iteracions sobre aquesta són possibles i així ho hem reflectit en les propostes de projecte per futurs estudiants. Per tant, tot i no resultar una funcionalitat tan satisfactòria com podríem haver anticipat, sí que compleix amb el rol principal que ens havíem marcat per ella, que era exposar el potencial emmascarat per les dades de l'API.

    Finalment, la funcionalitat d'evolució d'esdeveniments, és, des del meu punt de vista, la funcionalitat que permetria explotar de forma més eficaç, les dades emmagatzemades per l'API i produir resultats més satisfactoris, d'haver-se comportat com ens esperàvem. De totes maneres, considero que aquesta funcionalitat representa un bon punt de partida, per comparar els registres de diferents països o plantejar la viabilitat de certs projectes.

    Pel que fa al procés de certificació, a pesar de tots els entrebancs que es van patir intentant iniciar-lo, des del moment que es va aconseguir contactar a l'organització per correu electrònic, tot va avançar amb més fluïdesa.

    Al final, vaig quedar sorprès per la simplicitat del procés, doncs durant la mateixa vídeo-conferència, que va ser realitzada amb el mànager del procés de certificació, vam obtenir un accés a producció limitat, mentre que l'estudi tècnic de l'aplicació, començaria a succeir en paral·lel.

    De fet, descobrir l'existència d'aquest accés a producció limitat, que bàsicament significa que no es pot fer pública l'aplicació (però pot ser utilitzada pels desenvolupadors i fins a 100 persones més), resulta un dels millors descobriments d'aquest projecte, doncs convida als estudiants a realitzar projectes amb l'API de FamilySearch, sense gaire temor a no obtenir accés a producció.

    Aquest acord d'ús limitat, implica que el procés de certificació pot començar-se molt abans del que ho hem fet nosaltres, doncs l'existència d'un prototip funcional bàsic, podria ser suficient de cara a obtenir-lo i el producte podria acabar de desenvolupar-se, des d'aquesta posició.

    Vist amb perspectiva, és una pena que el procés de certificació s'hagi allargat fins a principis de setembre, doncs de no haver petit els endarreriments a l'inici del procés, haguéssim disposat de més temps per estudiar les dades de producció i el comportament real del SDK de Javascript. Tanmateix, aquest no havia estat mai un objectiu del projecte, sinó feina extra que ens agradaria haver pogut realitzar, pel fet d'arribar més lluny del que ens esperàvem.

    Així doncs, podem concloure que en el bloc pràctic del projecte, s'ha aconseguit crear un conjunt de funcionalitats, que permeten explorar les dades contingudes per l'API de FamilySearch i comprendre com  explotar el potencial d'aquesta, assolint així, el principal objectiu d'aquesta part del projecte.

    De forma addicional, no només s'ha aconseguit explorar el procés de certificació pels futurs estudiants, sinó que s'ha aconseguit un accés a producció, que ens ha permès jugar una mica amb les dades reals i explorar més a fons el Javascript SDK.

    Finalment, el fet que l'organització FamilySearch vulgui publicar el codi de la nostra aplicació, al costat de les aplicacions d'exemple oficials, perquè més desenvolupadors puguin contribuir a millorar-la o utilitzar-la, per comprendre els diferents usos que se li poden donar a la seva API, certifica que hem assolit els objectius didàctics d'aquesta secció del projecte, de forma més satisfactòria del que personalment m'esperava.
