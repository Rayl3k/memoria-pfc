\section{Introducció}

    \paragraph{}
    Aquest projecte neix de les conversacions amb Enric Mayol mentre explorava di\-fe\-rents possibilitats pel projecte final de carrera. L’Enric em va introduir l'organització FamilySearch i l'existència de la seva \gls{API}, encarregada de gestionar l’accés a les dades genealògiques.

    \gls{FamilySearch} és una organització sense ànim de lucre destinada a connectar famílies a través de diferents generacions. La seva visió com a col·lectiu és ajudar a les persones a crear un vincle amb els seus avantpassats, com a eina per poder comprendre millor qui són, crear un sentiment de família i teixir un pont entre el passat i futur.

    Com s'ha esmentat, les dades emmagatzemades per l'organització i accessibles a través de l'\gls{API} són principalment de caràcter genealògic. En concret, es disposa d'una col·lecció de persones de les quals se'n coneix, informació personal i esdeveniments rellevants en el transcurs de la seva vida, com podria ser per exemple les dades relatives al seu naixement i les seves relacions amb altres persones o el que és el mateix, el seu arbre genealògic.

    El projecte gira entorn aquesta \gls{API} i ha estat dividit en tres grans blocs o seccions.

    Per començar, la realització d'un estudi profund de l'\gls{API}.\ Això significa comprendre quines són les petites peces d’informació realment disponibles i com estan relacionades entre elles. D'aquesta forma, també s'ajudarà als futurs estudiants, interessats a utilitzar aquesta API, a comprendre-la i començar a utilitzar-la amb molta més facilitat.

    Un dels reptes d'aquesta primera tasca era proporcionar una documentació clara i completa, tot posant ordre a la caòtica organització de la informació per part de FamilySearch i fer un triatge de quina informació estava desfasada, quina era vigent i quina no es trobava encara documentada.

    En segon lloc i com un dels tres blocs principals del projecte, utilitzant el conei\-xe\-ment adquirit durant l’estudi de l'\gls{API}, així com les oportunitats i limitacions imposades per la plataforma, conegudes durant la implementació dels exemples, plantejar un conjunt de propostes de projecte que puguin servir a futurs estudiants, com a suport i inspiració.

    Finalment, l’últim bloc del projecte, consisteix a implementar una aplicació que interactuí amb l'\gls{API} de FamilySearch a través de diferents exemples i que serviran de prova pilot per facilitar l'observació i comprensió del potencial de l'\gls{API}, exposar-ne la informació emmagatzemada i oferir idees de com encarar-ne l'explotació.
