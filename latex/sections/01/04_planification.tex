\section{Planificació}

    \paragraph{}
    L'objectiu d'aquest apartat de la memòria és presentar les diferents planificacions que s'han portat a terme per encarar el projecte, en cada una de les convocatòries matriculades.

    \subsection{Planificació Febrer del 2014 -\ Juliol 2014}

        \paragraph{}
        Aquest projecte va ser matriculat per primera vegada al Febrer del 2014 amb la intenció de presentar-lo a principis de juliol del mateix any. La idea inicial era aprofitar el mes de gener per avançar feina i disposar així d’un total de sis mesos per realitzar el projecte.

        Al març del 2014 vaig començar a treballar a jornada completa i el projecte va deixar d’avançar a la velocitat esperada. Es van començar a patir forts endarreriments sobre la planificació original, fins al punt que el projecte va quedar completament aturat. La figura~\ref{fig:firstPlan} mostra en línies generals la planificació que s’hagués volgut portar a terme en cas de normalitat i ressaltat en vermell la part que es va veure interrompuda.

        \begin{figure}[h]
                \includegraphics[width=\linewidth]{01/firstPlan}
                \centering
                \caption{Planificació original \emph{Febrer 2014 -\ Juliol 2014}.\label{fig:firstPlan}}
        \end{figure}

        La falta de temps per realitzar un projecte acceptable, conjuntament a la poca capacitat de maniobra entre els mesos de març i juliol, va provocar que es descartés la possibilitat de presentar el projecte durant la convocatòria inicialment prevista.

    \subsection{Planificació Febrer del 2016 -\ Setembre 2016}\label{sec:thePlan}

        \paragraph{}
        A  conseqüència de l’extinció del pla d’enginyeries 2003, el projecte es torna a matricular al Febrer del 2016, tenint en consideració que s'hauria de començar pràcticament de zero.

        Donada la diferència de temps entre la primera inscripció i la segona, l’\gls{API} de FamilySearch s'havia vist sotmesa a grans canvis i la major part del material estudiat i coneixements tècnics adquirits fa dos anys, quedaven completament antiquats.

        A pesar de matricular el projecte a mitjans de Febrer, es coneixia que aquest no podria ser començat amb agilitat fins a principis d’abril a causa d'una situació excepcional en l'àmbit laboral. Gràcies a la disponibilitat d’una pròrroga extraordinària, que permetia estendre el període d'entrega fins a finals de setembre, la finestra de temps disponible per completar el projecte rondava els cinc o sis mesos.

        Cal tenir en compte que la disponibilitat horària en el dia a dia per treballar en el projecte, era molt reduïda i en conseqüència, realitzar una bona planificació resultava essencial si es volien evitar els mateixos problemes que van provocar l’abandonament del projecte en el seu primer intent.

        Tant en la figura~\ref{fig:actualPlan}, com en les seccions que segueixen a continuació, expliquem com va ser planificada i executada la feina entre els mesos d'abril i setembre.

        Els requadres més foscs de la figura, representen l'activitat principal en la qual s'ha invertit la major part del temps disponible en el període, mentre que els requadres més clars, representen activitats secundàries que s'han anat realitzant de forma paral·lela.

        \begin{figure}[h]
            \includegraphics[scale=0.5, angle=90]{01/actualPlan}
            \centering
            \caption{Planificació final \emph{Febrer 2016 -\ Setembre 2016}.\label{fig:actualPlan}}
        \end{figure}

        \subsubsection{Segona quinzena de març}

            \paragraph{}
            En aquest petit període de temps es va realitzar el primer estudi superficial sobre l'\gls{API} de FamilySearch amb la finalitat d’observar quins canvis s'havien produït durant els darrers dos anys i com aquests podien afectar o modificar la proposta inicial inscrita del projecte.

            L'objectiu d'aquesta repassada ràpida era proporcionar una visió global sobre certes limitacions que poguessin afectar el desenvolupament del projecte i ens permetés elaborar una planificació coherent de com afrontar i estructurar la feina a realitzar.

        \subsubsection{Primera quinzena d'abril}

            \paragraph{}
            Tot i que l’estudi sobre la informació disponible a través de l'\gls{API} es trobava en els seus inicis, es va aprofitar aquesta quinzena per decidir quina mena d’aplicació volíem implementar. Aquesta decisió obriria pas a la recerca i estudi de quines tecnologies serien més adients de cara a la implementació dels exemples i les comunicacions amb l'\gls{API} de FamilySearch.

            També s'aprofitaria aquesta quinzena per familiaritzant-nos amb la diferent do\-cu\-men\-ta\-ció disponible sobre l'\gls{API} de FamilySearch i plantejar-ne l’ordre d'estudi.

        \subsubsection{Segona quinzena d'abril -\ Finals de maig}

            \paragraph{}
            Aquest període inicial del projecte resultava crucial de cara a incorporar les eines necessàries al nostre coneixement, que ens permetrien completar un dels objectius principals del projecte durant els mesos següents.

            Així doncs, l'objectiu era el de detallar i estudiar, tots els aspectes referents a la part tècnica de l’aplicació. Escollir de forma correcta era indispensable si volíem evitar tancar-nos portes abans de començar o assegurar-nos de què utilitzàvem eines eficients que oferien un bon balanç entre esforç i qualitat. Els punts coberts durant aquesta fase van ser:

            \begin{itemize}
                \item Esbrinar els components principals, que conformen una pàgina web avui en dia i com interactuen entre ells.
                \item Conèixer les diferents tecnologies disponibles per cada un d’aquests components.
                \item Escollir del grup de tecnologies estudiat, les més adients per fer front als objectius del projecte i estudiar-les a fons.
            \end{itemize}

            \paragraph{}
            Un segon objectiu d’aquesta fase consistia a estudiar en profunditat l'\gls{API} de Family\-Search, amb l'objectiu de comprendre amb precisió quina informació contenia aquesta, quina mena de projectes es podria arribar a realitzar i quina no. Aquest exercici ens ajudaria a comprendre quines idees tindria més sentit implementar en la prova pilot, per demostrar la potencialitat i abast de l'\gls{API}.

            Aquesta tasca, de fet, s'allargaria fins finals del mes de juny.

        \subsubsection{Mesos de juny i juliol}

            \paragraph{}
            La major part de l’esforç durant aquests dos mesos es concentraria en la implementació de l’aplicació web. Es va decidir prioritzar aquesta tasca per sobre de la generació d'idees i l'estudi dels detalls més particulars de l'API, per dos motius.

            El primer, desenvolupar l’aplicació ens ajudaria a comprendre de forma més concisa tant la informació disponible, com les diferents relacions entre els objectes emmagatzemats. En segon lloc, acabar de desenvolupar l'aplicació al més aviat possible, ens oferia la possibilitat d’aplicar pel procés de certificació i en conseqüència, investigar en què consistia i intentar aconseguir accés a les dades de producció.

            A pesar dels avantatges que oferia començar pel desenvolupament de l'aplicació, en contrapartida, també significava tirar endavant una part important del projecte amb el risc de no haver arribat a conèixer la totalitat de l'abast de l'\gls{API}.

            Durant aquest període, també es van anar realitzant de forma secundària i per necessitat, petits estudis sobre l’estructura de l’API i estudis més específics sobre les tecnologies utilitzades, per tal de poder implementar i finalitzar, certes funcionalitats de l’aplicació web.

        \subsubsection{Mes d'agost}

            \paragraph{}
            L'objectiu principal del mes d'agost era fer front a aquella part del projecte que havia quedat oblidada fins aquest moment. En concret:

            \begin{itemize}
                \item Redactar la memòria del projecte.
                \item Detallar les diferents propostes de projecte pels futurs estudiants.
                \item Implementar la major part del contingut estàtic de l'aplicació web.
                \item Iniciar el procés de certificació amb FamilySearch.
            \end{itemize}

        \subsubsection{Mes de setembre}

            \paragraph{}
            El mes de setembre s'utilitzaria com a marge de maniobra per acabar de tancar aquelles tasques del projecte que poguessin estar sotmeses a petits retards. Segurament, la redacció de la memòria i petits retocs en l'aplicació web.

            Durant aquest període també s'aprofitaria per acabar de tancar el procés de certificació amb FamilySearch, que ens permetria obtenir accés a les dades de producció i tancar les diferents conversacions obertes per la publicació del projecte en la seva pàgina web.

            Finalment, durant la part final del mes, un cop el projecte es trobés entregat, aprofitaríem per preparar la defensa del projecte.
